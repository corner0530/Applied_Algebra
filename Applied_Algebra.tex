\documentclass[autodetect-engine,dvipdfmx-if-dvi,ja=standard,a4j]{bxjsarticle}

\usepackage{bm} % ベクトル
\usepackage{array} % 数式モードの表
\usepackage{tabularx} % 表
\usepackage{graphicx} % 画像の挿入など
\usepackage{longtable} % 改ページ可能な表
%\usepackage{mathcomp} % 数式モードのテキストシンボル
\usepackage{amsmath,amssymb} % 数式
\usepackage{xcolor} % 色
\usepackage{tcolorbox} % 色付きの枠
\usepackage{wrapfig} % 図に対して文書の回り込み
\usepackage{amsthm} % 定理環境
%\usepackage{newtxtext,newtxmath} % Times フォント
\usepackage{mathrsfs} % 花文字
\usepackage{empheq} % 連立方程式
\usepackage{framed} % 改ページ可能な表
\usepackage{tikz} % 図を書く
\usepackage{braket} % ブラケット記法
\usepackage{physics} % 数式を簡単に書く
\usepackage{listings,jvlisting} % ソースコードとか
\usepackage{comment} % コメント
\usepackage{bussproofs} % 証明図
\usepackage{pxrubrica} % ルビ
\usepackage{ulem} % 取り消し線
\usepackage{geometry} % 余白
\usepackage{ascmac} % 別行で文書を囲む
\usepackage{fancybox} % 行中で文書を囲む
\usepackage{tablefootnote} % 表中で脚注

%\geometry{top=30truemm,bottom=30truemm,inner=25truemm,outer=25truemm} % 余白調整

\lstset{
  %	\begin{lstlisting}[caption=hoge,label=fuga]
  %		ソースコード
  %	\end{lstlisting}
  %language = {C++},
  %backgroundcolor={\color[gray]{.85}},
  basicstyle={\ttfamily},
  identifierstyle={\small},
  commentstyle={\small\ttfamily},
  keywordstyle={\small\bfseries},
  ndkeywordstyle={\small},
  stringstyle={\small\ttfamily},
  frame={tb},
  breaklines=true,
  columns=[l]{fullflexible},
  numbers=left,
  xrightmargin=0\zw,
  xleftmargin=3\zw,
  numberstyle={\scriptsize},
  stepnumber=1,
  numbersep=1\zw,
  lineskip=-0.5ex
}
%\renewcommand{\lstlistingname}{Program}

\newtheoremstyle{mystyle1}% スタイル名
  {}% 上部スペース
  {}% 下部スペース
  {\normalfont}% 本文フォント(\itshape:斜体,\bfseries:太字,\normalfont:普通
  {}% 1行目のインデント量(\parindent:普通の段落)
  {\bfseries\sffamily}% 見出しフォント
  {}% 見出し後の句読点
  { }% 見出し後のスペース(\newline:改行)
  {\thmname{#1}\thmnumber{#2}\thmnote{\,(#3)\\}}% 見出しの書式
\theoremstyle{mystyle1}
\newtheorem{dfn}{定義}[section]
\newtheorem{thm}[dfn]{定理}
\newtheorem{axi}[dfn]{公理}
\newtheorem{cor}[dfn]{系}
\newtheorem{prop}[dfn]{命題}
\newtheorem{lem}[dfn]{補題}
\newtheorem{ex}[dfn]{例題}

\newtheoremstyle{mystyle2}% スタイル名
  {}% 上部スペース
  {}% 下部スペース
  {\normalfont}% 本文フォント(\itshape:斜体,\bfseries:太字,\normalfont:普通
  {}% 1行目のインデント量(\parindent:普通の段落)
  {\bfseries\sffamily}% 見出しフォント
  {}% 見出し後の句読点
  { }% 見出し後のスペース(\newline:改行)
  {\thmname{#1}\thmnote{:\,#3\\}}% 見出しの書式
\theoremstyle{mystyle2}
\newtheorem{dfn*}{定義}[section]
\newtheorem{thm*}[dfn*]{定理}
\newtheorem{axi*}[dfn*]{公理}
\newtheorem{cor*}[dfn*]{系}
\newtheorem{prop*}[dfn*]{命題}
\newtheorem{lem*}[dfn*]{補題}
\newtheorem{ex*}[dfn*]{例題}
\newtheorem{example*}[dfn*]{例}

\makeatletter
\renewenvironment{proof}[1][\proofname]{\par
  \pushQED{\qed}%
  \normalfont \topsep6\p@\@plus6\p@\relax
  \trivlist
  \item\relax
  {\bfseries % ここを変えた
  #1\@addpunct{.}}\hspace\labelsep\ignorespaces
}{%
  \popQED\endtrivlist\@endpefalse
}
\makeatother

\makeatletter
% インライン数式でカンマの後でも改行できるように
\def\old@comma{,}
\catcode`\,=13
\def,{%
  \ifmmode%
    \old@comma\discretionary{}{}{}%
  \else%
    \old@comma%
  \fi%
}
\makeatother

\newcommand{\redunderline}[1]{{\color{red} \underline{{\color{black} #1}}}}
\newcommand{\bbZ}{\mathbb{Z}}
\newcommand{\bbQ}{\mathbb{Q}}
\newcommand{\bbR}{\mathbb{R}}
\newcommand{\bbC}{\mathbb{C}}

% tlmgr update --self --all でアップデートすること

\begin{document}

\title{応用代数学}
\author{}
\date{\today 現在}
\maketitle

\setcounter{section}{-1}

\section{準備}
	集合$X,Y,Z$に対して
	\begin{itemize}
		\item $Y\backslash X$:$X,Y\subset Z$に対して,$X$に属さない$Y$の元のなす集合\[Y\backslash X=\qty\{y\mid y\in Y\text{\,かつ\,}y\not\in X\}\]
		\item $X\sqcup Y$:集合$X$と$Y$の直和(direct sum)あるいは分割和(disjoint sum)を表し,互いに交わらない二つの集合$X$と$Y$の和集合\[A\sqcup B\Leftrightarrow A\cup B\text{\,for\,}A\cap B=\emptyset\]
		\item (有限)集合$X$の元の個数を$\sharp X$\[\sharp\qty\{a,b,c\}=3\]
	\end{itemize}

	\underline{写像} 集合$X$から集合$Y$への写像$f$とは,$X$の各元$x$に対して$Y$の元$y=f\qty(x)$を対応させる規則
	\[f:X\to Y; x\mapsto y=f\qty(x)\]
	\begin{itemize}
		\item $f$による$A\subset X$の像(image)\[f\qty(A)=\qty{f\qty(a)\mid a\in A}\subset Y\]
		\item $f$による$B\subset Y$の逆像あるいは$f$による引き渡し\[f^{-1}\qty(B)=\qty{a\in X\mid f\qty(a)\in B}\]
		\item $f:X\to Y, g:Y\to Z$において,合成$g\circ f$
			\begin{align*}
				\begin{array}{ccccc}
					g\circ f: & X                     & \longrightarrow & Z                     &                                  \\
					          & \rotatebox{90}{$\in$} &                 & \rotatebox{90}{$\in$} &                                  \\
					          & x                     & \longmapsto     & z                     & =g\circ f\qty(x)=g\qty(f\qty(x))
				\end{array}
			\end{align*}
	\end{itemize}

	\subsection{集合の分割と同値関係}
		一般に集合$S$を交わりのない(空でない)集合の和で表すことを\redunderline{類別(classification)}という%undlineは赤線
		\[S=A_1\sqcup A_2\sqcup\cdots\]
		各部分集合$A_j$をこの類別における\redunderline{類}という

		全ての類から一つずつ元を選んできて得られる集合$R$を\redunderline{完全代表系}という.集合$R$の元を\redunderline{代表元}という.
		このとき,任意の$x\in S$に対して,集合$S$から集合族$\qty{A_1,A_2,\cdots}$への\redunderline{自然な全射写像}$x\mapsto A_j\qty(\ni x)$が得られる.

		\begin{dfn}[同値関係]
			集合$S$上の同値関係$\sim$$\Leftrightarrow$集合$S$上の二項関係$\sim$が任意の$a,b,c\in S$に対し,
			\begin{itemize}
				\item[反射則] $a\sim a$
				\item[対称則] $a\sim b \Rightarrow b\sim a$
				\item[推移則] $a\sim b\text{かつ}b\sim c\Rightarrow a\sim c$
			\end{itemize}
		\end{dfn}

		\begin{dfn}[同値類]
			集合$S$に同値関係$\sim$が存在するとき,$S$の各元$a$に対して定まる空でない部分集合
			\[C_a=\qty{x\mid x\in S,x\sim a}\]
			を$a$の同値類(equivalence class)という.
		\end{dfn}

		\begin{dfn}[商集合]
			同値関係$\sim$による同値類の集合を$S$の$\sim$による商集合(quotient set)といい$S/\sim$で表す
			\[S/\sim\,=\qty{C_a\mid a\in S}\]
		\end{dfn}

		\begin{dfn}[同値類別]
			同値関係による集合の分割.同値関係$\sim$による同値類別
			\[S=\bigsqcup_{a\in R\subset S}C_a\]
			ここで$S$の部分集合$R$は全ての同値類の代表元の集合であり,完全代表系である.
			このとき,自然な全射$\pi:S\to S/\sim;x\mapsto C_x$が存在する.
			さらに全単射写像$R\to S/\sim$も存在する.
		\end{dfn}
		\begin{screen}
			\[C_a\cap C_b\neq\emptyset\Rightarrow C_a=C_b\]
			が成立する(同値関係の性質から直ちに示される)
		\end{screen}

		\begin{example*}
			$S$:ある学校の学生全体の集合.$a,b\in S$の間の関係「クラスメートである」は同値関係.
			この関係を$\sim^\prime$で表すと,$a\sim^\prime b$は「$a$は$b$とクラスメートである」の意.このとき$C_a=\qty{x\in S\mid x\sim a}$は「$a$さんのクラスメート($a$さんを含む)の全体」$\leftrightarrow$ $a$さんの属するクラス
		\end{example*}

\section{群}
	\begin{dfn}[二項演算]
		一般に,集合$M$の2つの元$x,y$に対してただ一つの元$\mu\qty(x,y)\in M$が対応しているとき,$\mu$を$M$上の二項演算という.すなわち,\[\mu:M\times M\to M\]
		ここで記号$\mu$は省略して,$\mu\qty(x,y)$を$x\cdot y, x\circ y,$あるいは$xy$などと書く(誤解のない限り).
	\end{dfn}
	\begin{dfn}[群]
		群(group)$G$とは,以下の規則を満たす二項演算$\mu$をもつ集合のこと.
		\[\mu:G\times G\to G\]
		厳密には$\qty(G,\mu)$のことを群と呼ぶが群$G$などと省略することが多い.
		任意の$x,y,z\in G$に対して成立:
		\begin{enumerate}
			\item 結合法則\[\mu\qty(\mu\qty(x,y),z)=\mu\qty(x,\mu(y,z))\]
			\item 単位元$e\in G$の存在.\[\mu\qty(e,x)=\mu\qty(x,e)=x\]
			\item 逆元の存在.\[\mu\qty(x,x^\prime)=\mu\qty(x^\prime,x)=e\]なる$x^\prime\in G$\footnote{あとで$x$の逆元$x^\prime$は唯一に定まることを示す.$x^\prime$を$x^{-1}$と書くことが多い}
		\end{enumerate}
	\end{dfn}
	\begin{dfn}
		\redunderline{群$G$の位数(order)}:$G$に含まれる元の個数を表し,$\qty|G|$と書く.
		位数が有限のとき,\redunderline{有限群}という.有限群でないとき,\redunderline{無限群}という.
	\end{dfn}
	一般に群では($\mu\qty(x,y)\neq\mu\qty(y,x)$)
	\[x\circ y\neq y\circ x\quad\qty(\text{交換関係が必ずしも成立しない})\]
	\begin{dfn}
		$G$の任意の元について,$xy=yx$が成り立つとき可換群(Abel群)という.可換群の演算記号を加法的に$x+y$と書くとき,加法群と呼び,このとき加法に関する単位元を$0$と表し,零元と呼ぶことが多い($x$の逆元は$-x$).
	\end{dfn}
	\subsection{群の基本的な性質}
		\begin{prop}
			単位元$e$は存在すればただ1つ.逆元$x^\prime\neq x$に対してただ1つに定まる(通常,$x^\prime$を$x^{-1}$と書く.)
		\end{prop}
		\begin{proof}[単位元の一意性]
			\[e=e\circ e^\prime=e^\prime\]
		\end{proof}
		\begin{proof}[逆元の一意性]
			\[x^\prime=x^\prime\circ e=x^\prime\circ\qty(x\circ x^{\prime\prime})=\qty(x^\prime\circ x)\circ x^{\prime\prime}=e\circ x^{\prime\prime}=x^{\prime\prime}\]
		\end{proof}
		\begin{prop}[簡約律]
			$\qty(G,\circ)$が群のとき,$x,y,z\in G$に対して,
			\[x\neq y\Rightarrow z\circ x\neq z\circ y,\quad x\circ z\neq y\circ z\]
		\end{prop}
		\begin{proof}
			$z\circ x=z\circ y$とすると
			\begin{align*}
				x=e\circ x & =\qty(z^\prime\circ z)\circ x=z^\prime\circ \qty(z\circ x) \\
				           & =z^\prime\circ \qty(z\circ y)=\qty(z^\prime\circ z)\circ y \\
				           & =e\circ y = y
			\end{align*}
			これは$x\neq y$に矛盾.$x\circ z\neq y\circ z$も同様に示される.
		\end{proof}
		\begin{cor}[組み替え定理]
			位数$n$の群$G$に任意の元$x\in G$をかけて得られる集合を$G^\prime=xG$とする.$G=\qty{y_k\mid k=1,2,\cdots,n}$と$G^\prime=\qty{xy_k\mid k=1,2,\cdots,n}$の間には全単射の写像が存在し,$G$と$G^\prime$は対等である.
		\end{cor}
		\begin{example*}
			$\bbZ,\bbQ,\bbR,\bbC$: 通常の足し算を群演算として加法群.単位元$0$,$x$の逆元は$-x$.$K\in\qty{\bbQ,\bbR,\bbC}$に対して,$K^\times=K\backslash\qty{0}$:通常の掛け算を群演算とした乗法群.単位元$1$,$x$の逆元は$\frac{1}{x}=x^{-1}$.
		\end{example*}
		まず,位数の小さな群について考える.「有限集合$G$に対して,$\qty(G,\circ)$が群となるよう二項演算$\circ:G\times G\to G$を与える」
		\begin{itemize}
			\item 位数1の群$G_1=\qty{e}$の群積表 \begin{table}[h]
					\centering
					\caption{自明な群(trivial group)}
					\label{tabl:g1}
					\begin{tabular}{c|c}
						$G_1$  & $e$           \\\hline
						$e$    & $e\circ e=e$
					\end{tabular}
				\end{table}
			\item 位数2の群$G_2=\qty{e,a}$\begin{table}[h]
					\centering
					\caption{}
					\label{tabl:g2}
					\begin{tabular}{c|cc}
						$G_2$  & $e$           & $a$                                              \\\hline
						$e$    & $e\circ e=e$  & $e\circ a = a$                                   \\
						$a$    & $a\circ e=a$  & $a\circ a = e$\tablefootnote{$a$でないことは簡約律から従う.}
					\end{tabular}
				\end{table}
			\item 位数3の群$G_3=\qty{e,a,b}$\begin{table}[h]
					\centering
					\caption{}
					\label{tabl:g3}
					\begin{tabular}{c|ccc}
						$G_3$  & $e$  & $a$                                           & $b$  \\\hline
						$e$    & $e$  & $a$                                           & $b$  \\
						$a$    & $a$  & $b$\tablefootnote{$a\circ b$が$b$だと簡約律を満たさない}  & $e$  \\
						$b$    & $b$  & $e$                                           & $a$
					\end{tabular}
				\end{table}
			\item 位数4の群$G_4^{\qty(1)}=\qty{e,a,b,c},\circ_1$,$G_4^{\qty(2)}=\qty{e,a,b,c},\circ_2$
		\end{itemize}
		群積表では\\
		\centerline{元は各行,各列においてそれぞれ1回のみ現れる.(全単射)}
		について確かめることができる.特に,対角線に対して対称な場合,可換群,非対称な場合,非可換群.
		\subsubsection{群の例}
			\begin{itemize}
				\item 巡回群$C_n$
					\begin{dfn}
						一つの元のべきで群の全ての元が表示できるとき,この群を\redunderline{巡回群(cyclic group)}という.位数$n$の巡回群を$C_n$と書く.
					\end{dfn}
					$C_1=\qty{e},C_2=\qty{e,c}=\braket{c}{c^2=e}, C_3=\qty{e,c,c^2}=\braket{c}{c^3=e},C_4,\cdots,C_n=\qty{e,c,\cdots,c^{n-1}}=\braket{c}{c^n=e},\cdots,C_\infty$において指数法則$c^mc^n=c^{m+n}$が成り立ち,加法群$\bbZ$と同一視することができる.$C_\infty=\qty{e,c,c^{-1},c^2,c^{-2},\cdots}$
				\item 二面体群$D_n$
					\begin{dfn}
						位数$2n$の二面体群(dihedral group)$D_n$
						\begin{align*}
							D_n       & =\qty{e,a,a^2,\cdots,a^{n-1},b,ab,a^2b,\cdots,a^{n-1}b} \\
							          & =\braket{a,b}{e=a^n=b^2=abab}                           \\
							\qty|D_n| & =2n.
						\end{align*}
						二面体群の表記の1つに,任意の元が2つの元$a,b$のいくつかの積で表示するものがある.このとき,$D_n$は集合$\qty{a,b}\subset D_n$によって生成されるといい,$\qty{a,b}$を\redunderline{生成系},生成系の元を\redunderline{生成元}という.

						また,生成元の間の任意の関係式が還元される関係式のことを\redunderline{基本関係式}という.\footnote{$D_n$では,$e=a^n=b^2=abab$}
					\end{dfn}
					\begin{itemize}
						\item 群積表で構成した群において,
							\[G_1=C_1,G_2=C_2,G_3=C_3,G_4^{\qty(1)}=C_4\]
						\item 巡回群でない最小位数の群$D_2=\qty(G_4^{\qty(2)})$
						\item 位数最小の非可換な群$D_3$
					\end{itemize}
					\begin{dfn}
						一般の群$G$において,元$x\in G$が生成する巡回群の位数を\redunderline{元$x$の位数}という\footnote{cf.群の位数}.$x^n=e$となる最小の$n>0$が$x$の位数(ない場合は,元$x$の位数は$\infty$として扱う).
					\end{dfn}
					\begin{example*}
						\[C_4=\qty{e,c,c^2,c^3}=\braket{c}{c^4=e}\]
						$c$の位数は4,$c^2$の位数は2,$\cdots$
					\end{example*}
				\item (行列群) $n$次正則行列の全体は,行列積について群をなす.単位元:\,$n$次単位行列,$x$の逆元は$x$の逆行列.\\
					\begin{itemize}
						\item 一般線形群\[GL_n\qty(\bbC)=\qty{g\in\text{Mat}\qty(n\times n;\bbC)\mid\det g\neq 0}\]
						\item 特殊線形群\[SL_n\qty(\bbC)=\qty{g\in GL_n\qty(\bbC)\mid\det g=1}\]
						\item 直交群\[O_n=\qty{g\in GL_n\qty(\bbR)\mid\,^t\!g\cdot g=1}\]
						\item ユニタリー群\[U_n=\qty{g\in GL_n\qty(\bbC)\mid\overline{^t\!g}\cdot g=1}\]
					\end{itemize}
			\end{itemize}
\end{document}
