\documentclass[autodetect-engine,dvipdfmx-if-dvi,ja=standard,a4j]{bxjsarticle}

\usepackage{bm} % ベクトル
\usepackage{array} % 数式モードの表
\usepackage{tabularx} % 表
\usepackage{graphicx} % 画像の挿入など
\usepackage{longtable} % 改ページ可能な表
%\usepackage{mathcomp} % 数式モードのテキストシンボル
\usepackage{amsmath,amssymb} % 数式
\usepackage{xcolor} % 色
\usepackage{tcolorbox} % 色付きの枠
\usepackage{wrapfig} % 図に対して文書の回り込み
\usepackage{amsthm} % 定理環境
%\usepackage{newtxtext,newtxmath} % Times フォント
\usepackage{mathrsfs} % 花文字
\usepackage{empheq} % 連立方程式
\usepackage{framed} % 改ページ可能な表
\usepackage{tikz} % 図を書く
\usepackage{tikz-3dplot} % tikz で 3D
\usepackage{braket} % ブラケット記法
\usepackage{physics} % 数式を簡単に書く
\usepackage{listings,jvlisting} % ソースコードとか
\usepackage{comment} % コメント
\usepackage{bussproofs} % 証明図
\usepackage{pxrubrica} % ルビ
\usepackage{ulem} % 取り消し線
\usepackage{geometry} % 余白
\usepackage{ascmac} % 別行で文書を囲む
\usepackage{fancybox} % 行中で文書を囲む
\usepackage{tablefootnote} % 表中で脚注

%\geometry{top=30truemm,bottom=30truemm,inner=25truemm,outer=25truemm} % 余白調整

\lstset{
  %	\begin{lstlisting}[caption=hoge,label=fuga]
  %		ソースコード
  %	\end{lstlisting}
  %language = {C++},
  %backgroundcolor={\color[gray]{.85}},
  basicstyle={\ttfamily},
  identifierstyle={\small},
  commentstyle={\small\ttfamily},
  keywordstyle={\small\bfseries},
  ndkeywordstyle={\small},
  stringstyle={\small\ttfamily},
  frame={tb},
  breaklines=true,
  columns=[l]{fullflexible},
  numbers=left,
  xrightmargin=0\zw,
  xleftmargin=3\zw,
  numberstyle={\scriptsize},
  stepnumber=1,
  numbersep=1\zw,
  lineskip=-0.5ex
}
%\renewcommand{\lstlistingname}{Program}

\newtheoremstyle{mystyle1}% スタイル名
  {}% 上部スペース
  {}% 下部スペース
  {\normalfont}% 本文フォント(\itshape:斜体,\bfseries:太字,\normalfont:普通
  {}% 1行目のインデント量(\parindent:普通の段落)
  {\bfseries\sffamily}% 見出しフォント
  {}% 見出し後の句読点
  { }% 見出し後のスペース(\newline:改行)
  {\thmname{#1}\thmnumber{#2}\thmnote{\,(#3)\\}}% 見出しの書式
\theoremstyle{mystyle1}
\newtheorem{dfn}{定義}[section]
\newtheorem{thm}[dfn]{定理}
\newtheorem{axi}[dfn]{公理}
\newtheorem{cor}[dfn]{系}
\newtheorem{prop}[dfn]{命題}
\newtheorem{lem}[dfn]{補題}
\newtheorem{ex}[dfn]{例}

\newtheoremstyle{mystyle2}% スタイル名
  {}% 上部スペース
  {}% 下部スペース
  {\normalfont}% 本文フォント(\itshape:斜体,\bfseries:太字,\normalfont:普通
  {}% 1行目のインデント量(\parindent:普通の段落)
  {\bfseries\sffamily}% 見出しフォント
  {}% 見出し後の句読点
  { }% 見出し後のスペース(\newline:改行)
  {\thmname{#1}\thmnote{:\,(#3)\\}}% 見出しの書式
\theoremstyle{mystyle2}
\newtheorem{dfn*}{定義}[section]
\newtheorem{thm*}[dfn*]{定理}
\newtheorem{axi*}[dfn*]{公理}
\newtheorem{cor*}[dfn*]{系}
\newtheorem{prop*}[dfn*]{命題}
\newtheorem{lem*}[dfn*]{補題}
\newtheorem{ex*}[dfn*]{例題}
\newtheorem{example*}[dfn*]{例}
\newtheorem{remark*}[dfn*]{Remark}
\newtheorem{note*}[dfn*]{Note}
\newtheorem{abbr*}[dfn*]{略証}

\makeatletter
\renewenvironment{proof}[1][\proofname]{\par
  \pushQED{\qed}%
  \normalfont \topsep6\p@\@plus6\p@\relax
  \trivlist
  \item\relax
  {\bfseries % ここを変えた
  #1\@addpunct{.}}\hspace\labelsep\ignorespaces
}{%
  \popQED\endtrivlist\@endpefalse
}
\makeatother

\makeatletter
% インライン数式でカンマの後でも改行できるように
\def\old@comma{,}
\catcode`\,=13
\def,{%
  \ifmmode%
    \old@comma\discretionary{}{}{}%
  \else%
    \old@comma%
  \fi%
}
\makeatother

\newcommand{\redunderline}[1]{{\color{red} \underline{{\color{black} #1}}}}
\newcommand{\blueunderline}[1]{{\color{blue} \underline{{\color{black} #1}}}}
\newcommand{\bbZ}{\ensuremath{\mathbb{Z}}}
\newcommand{\bbQ}{\ensuremath{\mathbb{Q}}}
\newcommand{\bbR}{\ensuremath{\mathbb{R}}}
\newcommand{\bbC}{\ensuremath{\mathbb{C}}}
\newcommand{\frakS}{\ensuremath{\mathfrak{S}}}
\newcommand{\frakA}{\ensuremath{\mathfrak{A}}}
\newcommand{\calH}{\ensuremath{\mathcal{H}}}
\newcommand{\Hom}{\ensuremath{\text{Hom}}\,}
\newcommand{\Ker}{\ensuremath{\text{Ker}}\,}
\newcommand{\End}{\ensuremath{\text{End}}\,}
\newcommand{\Aut}{\ensuremath{\text{Aut}}\,}
\newcommand{\sgn}{\ensuremath{\text{sgn}}\,}
\newcommand{\GL}{\ensuremath{\text{GL}}}
\newcommand{\PGL}{\ensuremath{\text{PGL}}}
\newcommand{\SL}{\ensuremath{\text{SL}}}
\newcommand{\PSL}{\ensuremath{\text{PSL}}}
\newcommand{\suchthat}{\ensuremath{\text{\quad s.t.\quad}}}

% tlmgr update --self --all でアップデートすること

\begin{document}

\title{応用代数学}
\author{}
\date{\today 現在}
\maketitle

\tableofcontents
\newpage

\setcounter{section}{-1}

\section{準備}
	集合$X,Y,Z$に対して
	\begin{itemize}
		\item $Y\backslash X$:$X,Y\subset Z$に対して,$X$に属さない$Y$の元のなす集合\[Y\backslash X=\qty{y\mid y\in Y\text{\,かつ\,}y\not\in X}\]
		\item $X\sqcup Y$:集合$X$と$Y$の直和(direct sum)あるいは分割和(disjoint sum)を表し,互いに交わらない二つの集合$X$と$Y$の和集合\[A\sqcup B\Leftrightarrow A\cup B\text{\quad for\quad}A\cap B=\emptyset\]
		\item (有限)集合$X$の元の個数を$\sharp X$\[\sharp\qty{a,b,c}=3\]
	\end{itemize}

	\underline{写像} 集合$X$から集合$Y$への写像$f$とは,$X$の各元$x$に対して$Y$の元$y=f\qty(x)$を対応させる規則
	\[f:X\to Y; x\mapsto y=f\qty(x)\]
	\begin{itemize}
		\item $f$による$A\subset X$の像(image)\[f\qty(A)=\qty{f\qty(a)\mid a\in A}\subset Y\]
		\item $f$による$B\subset Y$の逆像あるいは$f$による引き渡し\[f^{-1}\qty(B)=\qty{a\in X\mid f\qty(a)\in B}\]
		\item $f:X\to Y, g:Y\to Z$において,合成$g\circ f$
			\begin{align*}
				\begin{array}{ccccc}
					g\circ f: & X                     & \longrightarrow & Z                     &                                  \\
					          & \rotatebox{90}{$\in$} &                 & \rotatebox{90}{$\in$} &                                  \\
					          & x                     & \longmapsto     & z                     & =g\circ f\qty(x)=g\qty(f\qty(x))
				\end{array}
			\end{align*}
	\end{itemize}

	\subsection{集合の分割と同値関係}
		一般に集合$S$を交わりのない(空でない)集合の和で表すことを\redunderline{類別(classification)}という%undlineは赤線
		\[S=A_1\sqcup A_2\sqcup\cdots\]
		各部分集合$A_j$をこの類別における\redunderline{類}という

		全ての類から一つずつ元を選んできて得られる集合$R$を\redunderline{完全代表系}という.集合$R$の元を\redunderline{代表元}という.
		このとき,任意の$x\in S$に対して,集合$S$から集合族$\qty{A_1,A_2,\cdots}$への\redunderline{自然な全射写像}$x\mapsto A_j\qty(\ni x)$が得られる.

		\begin{dfn}[同値関係]
			集合$S$上の同値関係$\sim$$\Leftrightarrow$集合$S$上の二項関係$\sim$が任意の$a,b,c\in S$に対し,
			\begin{itemize}
				\item[反射則] $a\sim a$
				\item[対称則] $a\sim b \Rightarrow b\sim a$
				\item[推移則] $a\sim b\text{かつ}b\sim c\Rightarrow a\sim c$
			\end{itemize}
		\end{dfn}

		\begin{dfn}[同値類]
			集合$S$に同値関係$\sim$が存在するとき,$S$の各元$a$に対して定まる空でない部分集合
			\[C_a=\qty{x\mid x\in S,x\sim a}\]
			を$a$の同値類(equivalence class)という.
		\end{dfn}

		\begin{dfn}[商集合]
			同値関係$\sim$による同値類の集合を$S$の$\sim$による商集合(quotient set)といい$S/\sim$で表す
			\[S/\sim\,=\qty{C_a\mid a\in S}\]
		\end{dfn}

		\begin{dfn}[同値類別]
			同値関係による集合の分割.同値関係$\sim$による同値類別
			\[S=\bigsqcup_{a\in R\subset S}C_a\]
			ここで$S$の部分集合$R$は全ての同値類の代表元の集合であり,完全代表系である.
			このとき,自然な全射$\pi:S\to S/\sim;x\mapsto C_x$が存在する.
			さらに全単射写像$R\to S/\sim$も存在する.
		\end{dfn}
		\begin{screen}
			\[C_a\cap C_b\neq\emptyset\Rightarrow C_a=C_b\]
			が成立する(同値関係の性質から直ちに示される)
		\end{screen}

		\begin{example*}
			$S$:ある学校の学生全体の集合.$a,b\in S$の間の関係「クラスメートである」は同値関係.
			この関係を$\sim^\prime$で表すと,$a\sim^\prime b$は「$a$は$b$とクラスメートである」の意.このとき$C_a=\qty{x\in S\mid x\sim a}$は「$a$さんのクラスメート($a$さんを含む)の全体」$\leftrightarrow$ $a$さんの属するクラス
		\end{example*}

\section{群}
	\begin{dfn}[二項演算]
		一般に,集合$M$の2つの元$x,y$に対してただ一つの元$\mu\qty(x,y)\in M$が対応しているとき,$\mu$を$M$上の二項演算という.すなわち,\[\mu:M\times M\to M\]
		ここで記号$\mu$は省略して,$\mu\qty(x,y)$を$x\cdot y, x\circ y,$あるいは$xy$などと書く(誤解のない限り).
	\end{dfn}
	\begin{dfn}[群]
		群(group)$G$とは,以下の規則を満たす二項演算$\mu$をもつ集合のこと.
		\[\mu:G\times G\to G\]
		厳密には$\qty(G,\mu)$のことを群と呼ぶが群$G$などと省略することが多い.
		任意の$x,y,z\in G$に対して成立:
		\begin{enumerate}
			\item 結合法則\[\mu\qty(\mu\qty(x,y),z)=\mu\qty(x,\mu(y,z))\]
			\item 単位元$e\in G$の存在.\[\mu\qty(e,x)=\mu\qty(x,e)=x\]
			\item 逆元の存在.\[\mu\qty(x,x^\prime)=\mu\qty(x^\prime,x)=e\]なる$x^\prime\in G$\footnote{あとで$x$の逆元$x^\prime$は唯一に定まることを示す.$x^\prime$を$x^{-1}$と書くことが多い}
		\end{enumerate}
	\end{dfn}
	\begin{dfn}
		\redunderline{群$G$の位数(order)}:$G$に含まれる元の個数を表し,$\qty|G|$と書く.
		位数が有限のとき,\redunderline{有限群}という.有限群でないとき,\redunderline{無限群}という.
	\end{dfn}
	一般に群では($\mu\qty(x,y)\neq\mu\qty(y,x)$)
	\[x\circ y\neq y\circ x\quad\qty(\text{交換関係が必ずしも成立しない})\]
	\begin{dfn}
		$G$の任意の元について,$xy=yx$が成り立つとき可換群(Abel群)という.可換群の演算記号を加法的に$x+y$と書くとき,加法群と呼び,このとき加法に関する単位元を$0$と表し,零元と呼ぶことが多い($x$の逆元は$-x$).
	\end{dfn}
	\subsection{群の基本的な性質}
		\begin{prop}
			単位元$e$は存在すればただ1つ.逆元$x^\prime\neq x$に対してただ1つに定まる(通常,$x^\prime$を$x^{-1}$と書く.)
		\end{prop}
		\begin{proof}[単位元の一意性]
			\[e=e\circ e^\prime=e^\prime\]
		\end{proof}
		\begin{proof}[逆元の一意性]
			\[x^\prime=x^\prime\circ e=x^\prime\circ\qty(x\circ x^{\prime\prime})=\qty(x^\prime\circ x)\circ x^{\prime\prime}=e\circ x^{\prime\prime}=x^{\prime\prime}\]
		\end{proof}
		\begin{prop}[簡約律]
			$\qty(G,\circ)$が群のとき,$x,y,z\in G$に対して,
			\[x\neq y\Rightarrow z\circ x\neq z\circ y,\quad x\circ z\neq y\circ z\]
		\end{prop}
		\begin{proof}
			$z\circ x=z\circ y$とすると
			\begin{align*}
				x=e\circ x & =\qty(z^\prime\circ z)\circ x=z^\prime\circ \qty(z\circ x) \\
				           & =z^\prime\circ \qty(z\circ y)=\qty(z^\prime\circ z)\circ y \\
				           & =e\circ y = y
			\end{align*}
			これは$x\neq y$に矛盾.$x\circ z\neq y\circ z$も同様に示される.
		\end{proof}
		\begin{cor}[組み替え定理]
			位数$n$の群$G$に任意の元$x\in G$をかけて得られる集合を$G^\prime=xG$とする.$G=\qty{y_k\mid k=1,2,\cdots,n}$と$G^\prime=\qty{xy_k\mid k=1,2,\cdots,n}$の間には全単射の写像が存在し,$G$と$G^\prime$は対等である.
		\end{cor}
		\begin{example*}
			$\bbZ,\bbQ,\bbR,\bbC$: 通常の足し算を群演算として加法群.単位元$0$,$x$の逆元は$-x$.$K\in\qty{\bbQ,\bbR,\bbC}$に対して,$K^\times=K\backslash\qty{0}$:通常の掛け算を群演算とした乗法群.単位元$1$,$x$の逆元は$\frac{1}{x}=x^{-1}$.
		\end{example*}
		まず,位数の小さな群について考える.「有限集合$G$に対して,$\qty(G,\circ)$が群となるよう二項演算$\circ:G\times G\to G$を与える」
		\begin{itemize}
			\item 位数1の群$G_1=\qty{e}$の群積表 \begin{table}[h]
					\centering
					\caption{自明な群(trivial group)}
					\label{tabl:g1}
					\begin{tabular}{c|c}
						$G_1$  & $e$           \\\hline
						$e$    & $e\circ e=e$
					\end{tabular}
				\end{table}
			\item 位数2の群$G_2=\qty{e,a}$\begin{table}[h]
					\centering
					\caption{}
					\label{tabl:g2}
					\begin{tabular}{c|cc}
						$G_2$  & $e$           & $a$                                              \\\hline
						$e$    & $e\circ e=e$  & $e\circ a = a$                                   \\
						$a$    & $a\circ e=a$  & $a\circ a = e$\tablefootnote{$a$でないことは簡約律から従う.}
					\end{tabular}
				\end{table}
			\item 位数3の群$G_3=\qty{e,a,b}$\begin{table}[h]
					\centering
					\caption{}
					\label{tabl:g3}
					\begin{tabular}{c|ccc}
						$G_3$  & $e$  & $a$                                           & $b$  \\\hline
						$e$    & $e$  & $a$                                           & $b$  \\
						$a$    & $a$  & $b$\tablefootnote{$a\circ b$が$b$だと簡約律を満たさない}  & $e$  \\
						$b$    & $b$  & $e$                                           & $a$
					\end{tabular}
				\end{table}
			\item 位数4の群$G_4^{\qty(1)}=\qty{e,a,b,c},\circ_1$,$G_4^{\qty(2)}=\qty{e,a,b,c},\circ_2$
		\end{itemize}
		群積表では\\
		\centerline{元は各行,各列においてそれぞれ1回のみ現れる.(全単射)}
		について確かめることができる.特に,対角線に対して対称な場合,可換群,非対称な場合,非可換群.
	\subsection{群の例}
		\begin{itemize}
			\item 巡回群$C_n$
				\begin{dfn}
					一つの元のべきで群の全ての元が表示できるとき,この群を\redunderline{巡回群(cyclic group)}という.位数$n$の巡回群を$C_n$と書く.
				\end{dfn}
				$C_1=\qty{e},C_2=\qty{e,c}=\braket{c}{c^2=e}, C_3=\qty{e,c,c^2}=\braket{c}{c^3=e},C_4,\cdots,C_n=\qty{e,c,\cdots,c^{n-1}}=\braket{c}{c^n=e},\cdots,C_\infty$において指数法則$c^mc^n=c^{m+n}$が成り立ち,加法群$\bbZ$と同一視することができる.$C_\infty=\qty{e,c,c^{-1},c^2,c^{-2},\cdots}$
			\item 二面体群$D_n$
				\begin{dfn}
					位数$2n$の二面体群(dihedral group)$D_n$
					\begin{align*}
						D_n       & =\qty{e,a,a^2,\cdots,a^{n-1},b,ab,a^2b,\cdots,a^{n-1}b} \\
						          & =\braket{a,b}{e=a^n=b^2=abab}                           \\
						\qty|D_n| & =2n.
					\end{align*}
					二面体群の表記の1つに,任意の元が2つの元$a,b$のいくつかの積で表示するものがある.このとき,$D_n$は集合$\qty{a,b}\subset D_n$によって生成されるといい,$\qty{a,b}$を\redunderline{生成系},生成系の元を\redunderline{生成元}という.

					また,生成元の間の任意の関係式が還元される関係式のことを\redunderline{基本関係式}という.\footnote{$D_n$では,$e=a^n=b^2=abab$}
				\end{dfn}
				\begin{itemize}
					\item 群積表で構成した群において,
						\[G_1=C_1,G_2=C_2,G_3=C_3,G_4^{\qty(1)}=C_4\]
					\item 巡回群でない最小位数の群$D_2=\qty(G_4^{\qty(2)})$
					\item 位数最小の非可換な群$D_3$
				\end{itemize}
				\begin{dfn}
					一般の群$G$において,元$x\in G$が生成する巡回群の位数を\redunderline{元$x$の位数}という\footnote{cf.群の位数}.$x^n=e$となる最小の$n>0$が$x$の位数(ない場合は,元$x$の位数は$\infty$として扱う).
				\end{dfn}
				\begin{example*}
					\[C_4=\qty{e,c,c^2,c^3}=\braket{c}{c^4=e}\]
					$c$の位数は4,$c^2$の位数は2,$\cdots$
				\end{example*}
			\item (行列群) $n$次正則行列の全体は,行列積について群をなす.単位元:\,$n$次単位行列,$x$の逆元は$x$の逆行列.\\
				\begin{itemize}
					\item 一般線形群\[\GL_n\qty(\bbC)=\qty{g\in\text{Mat}\qty(n\times n;\bbC)\mid\det g\neq 0}\]
					\item 特殊線形群\[\SL_n\qty(\bbC)=\qty{g\in \GL_n\qty(\bbC)\mid\det g=1}\]
					\item 直交群\[O_n=\qty{g\in \GL_n\qty(\bbR)\mid\,^t\!g\cdot g=1}\]
					\item ユニタリー群\[U_n=\qty{g\in \GL_n\qty(\bbC)\mid\overline{^t\!g}\cdot g=1}\]
				\end{itemize}
		\end{itemize}
	\subsection{いろいろな代数系}
		群以外にさまざまな"代数"が考えられる.
		\begin{dfn}
			集合$M$上に二項演算$\mu:M\times M\to M$が定義されているとする.
			\begin{itemize}
				\item 結合法則を満たすとき,$M$は\redunderline{半群(semi group)}であるという.
					\[\mu\qty(\mu\qty(x,y),z)=\mu\qty(x,\mu\qty(y,z))\]
				\item 単位元(identity)$e\in M$を持つ半群を\redunderline{単位的半群}あるいは,\redunderline{モノイド(monoid)}という.$e$を$1_M,I_d$と表すこともある.
					\[\mu\qty(e,x)=\mu\qty(x,e)=x\]
				\item 任意の$x\in M$に対して
					\[\mu\qty(x,x^\prime)=\mu\qty(x^\prime,x)=e\]
					を満たす$x^\prime\in M$が存在するモノイドを\redunderline{群}という.
			\end{itemize}
		\end{dfn}
		さらに,環,体などの複雑な"代数系"もある.
		\begin{example*}[半群の例]%TODO: 直す
			$\qty(\bbR,\max)$: $x,y\in\bbR$に対して
			\[x\circ y=\max\qty(x,y)\]
			このとき,$\qty(\bbR,\max)$は半群.
			\[\max\qty(\max\qty(x,y),z)=\max(x,\max\qty(y,z))\]
			さらに形式的な単位元となる$-\infty$を考慮に入れる場合,$\qty(\bbR\cup\qty{-\infty},\max)$はモノイドとなる.逆元は存在しない($\cdots\max\qty(x,x^\prime)=-\infty$となる$x^\prime$が存在しない)
		\end{example*}
		\begin{dfn}
			モノイド$M$の元で逆元$x^\prime$が存在するものを\redunderline{単元(unit)}といい$M^\times$を$M$の単元のなす部分集合とする.このとき,$M^\prime$は群をなし,\redunderline{単元群(unit group)}と呼ばれる.
		\end{dfn}
		\begin{example*}
			$K\in\qty{\bbQ,\bbR,\bbC}$に対して,$K^\times=K\backslash\qty{0}$は積に関して単元群.
		\end{example*}
	\subsection{写像の合成と群}
		集合$X$から$X$自身への写像全体のなす集合を$M\qty(X)$とかく.$x\in X$および$f,g\in M\qty(X)$に対し,写像の合成$f\cdot g\qty(x)=f\qty(g\qty(x))$を$f\cdot g\in M\qty(X)$とかく.このとき
		\begin{enumerate}
			\item $f,g,h\in M\qty(X)$に対し,$\qty(f\cdot g)\cdot h=f\cdot\qty(g\cdot h)$
			\item $e=Id_{X}$を$X$の恒等写像とすると$f\cdot e=e\cdot f=f$
			\item $S\qty(X)=\qty{f\in M\qty(X)\mid f\text{は全単射}}$とおくと,$e\in S\qty(X)$で,$f,g\in S\qty(X)$ならば,$f\cdot g\in S\qty(X)$.また$f^{-1}$を$f\in S\qty(X)$の逆写像とすると
				\[f\cdot f^{-1}=f^{-1}\cdot f=e\]
		\end{enumerate}
		以上より,次の命題を得る.
		\begin{prop}
			任意の集合$X$に対し,$X$から$X$への写像の集合$M\qty(X)$はモノイドをなす.全単射のなす部分集合$S\qty(X)=M\qty(X)^\times$が単位元となる.$S\qty(X)$を$X$の\redunderline{対称群(symmetric group)}

			特に$X$が有限集合のとき$\qty(\sharp X=n<\infty)$,$S\qty(X)$を$n$次対称群といい,$S\qty(X)=S_n$あるいは$\frakS_n$とかく.$S_n$を$n$次置換群(permutation group)と呼ぶ場合もある.$S_n$の位数は$\qty|S_n|=n!$
		\end{prop}
		\begin{remark*}
			特に$X=\qty{1,2,\cdots,n}$のとき,$S_n=S\qty(X)$の元$\sigma$を1つ選べば,重複なしに$n$個の数字を並べたもの,つまり,$1,2,\cdots,n$の順列の1つが定まる.
			\[\sigma:X\overset{\text{1to1}}{\to}X;\quad j\mapsto\sigma\qty(j)\in X\]
			置換の記法
			$X=\qty{1,2,\cdots,n}=\qty{i_1,i_2,\cdots,i_n}=\qty{j_1,j_2,\cdots,j_n}$に対して,$\sigma=\mqty(i_1&i_2&\cdots&i_n\\j_1&j_2&\cdots&j_n)$は$\sigma\qty(i_k)=j_k\quad\qty(k=1,2,\cdots,n)$を定める.例えば,$X=\qty{1,2,3}$において,$\sigma=\mqty(1&2&3\\3&1&2)\in S_3$とすると,
			\[\sigma\qty(1)=3,\quad\sigma\qty(2)=1,\quad\sigma\qty(3)=2\]
			ここで,
			\[\sigma=\mqty(2&3&1\\1&2&3)=\mqty(1&3&2\\3&2&1)=\mqty(1&2&3\\3&1&2)\]
		\end{remark*}
		\begin{dfn}[交代群]
			$S_n$の元の中で偶数個の互換の積で表すことができる元の全体を$A_n$で表す.\footnote{ここで互換とは,2つの数字(文字)の入れ替え $i\leftrightarrow j$を表し,$\qty(i,j)$と表記する.}
			この$S_n$の部分集合$A_n$は,$S_n$の演算で群となり,\redunderline{$n$次交代群(alternating group)}と呼ばれる.
		\end{dfn}
		\begin{prop}
			$S_n=A_n\cup\qty(1,2)A_n$である($n\geq 2$).
			\[\qty|A_n|=\frac{\qty|S_n|}{2}=\frac{n!}{2}\]
		\end{prop}
		\begin{remark*}
			表示方法によって偶奇性が変わらないことを証明する必要
		\end{remark*}
		\subsubsection*{巡回置換の記号}
			\[\qty(j_1,j_2,\cdots,j_l)\coloneqq\mqty(j_1&j_2&\cdots&j_{l-1}&j_l\\j_2&j_3&\cdots&j_l&j_1)\]
			\begin{align*}
				S_3 & =\qty{\mqty(1                                                                                                                                                                               & 2 & 3 \\1&2&3),\mqty(1&2&3\\1&3&2),\mqty(1&2&3\\3&2&1),\mqty(1&2&3\\2&1&3),\mqty(1&2&3\\3&1&2),\mqty(1&2&3\\2&3&1)}\\
				    & =\qty{{\color{red} e},{\color{blue} \qty(2,3)},{\color{blue} \qty(1,3)},{\color{blue} \qty(1,2)},{\color{red} \qty(1,3,2)=\qty(2,3)\qty(1,2)},{\color{red} \qty(1,2,3)=\qty(1,2)\qty(2,3)}}         \\
				A_3 & =\qty{e,\qty(2,3)\qty(1,2),\qty(1,2)\qty(2,3)}
			\end{align*}
	\subsection{部分群}
		\begin{dfn}
			群$G$の空でない部分集合$H$が群$G$の演算で群となるとき,$H$を$G$の\redunderline{部分群(subgroup)}と呼び,$H\leqq G$と書くことがある.
		\end{dfn}
		\begin{remark*}
			群$G$の自明な部分群に,$G$の単位元のみからなる群と,$G$自身の2つある.
		\end{remark*}
		\begin{prop}
			群$G$の空でない部分集合$H$が
			\[x,y\in H\,\Rightarrow\,x\cdot y^{-1}\in H\]
			をみたすとき,$H$は$G$の部分群となる.$H$は単位元を含み,$G$の演算で群をなす.
		\end{prop}
		\begin{proof}
			$x\in H$とすると,$e=x\cdot x^{-1}\in H$.
			\[x\in H\,\Rightarrow\,x^{-1}=x^{-1}\cdot e\in H\]
			\[x,y\in H\,\Rightarrow\,x,y^{-1}\in H\,\Rightarrow\,xy\in H\]
			$H$は空でない部分群.
		\end{proof}
\section{準同型写像}
	準同形写像: 群の二項演算の代数的構造を保つ写像
	\subsubsection*{群の例}
		\begin{enumerate}
			\item[(1)] $G=\bbZ,\quad a\circ b=a+b+1,\quad\qty(G,\circ)$は群をなす.この群の単位元および$a$の逆元を求めよ
				\[a\circ e=e\circ a=a\Leftrightarrow a+e+1=e+a+1=a\Leftrightarrow e=-1\]
				\[a\circ a^{-1}=a^{-1}\circ a=e\Leftrightarrow a+a^{-1}+1=a^{-1}+a+1=-1\Leftrightarrow a^{-1}=-a-2\]
			\item[(2)] $G=\bbR\backslash\qty{-1},\quad a\circ b=a+b+ab,\quad\qty(G,\circ)$は群をなす.この群の単位元および$a$の逆元を求めよ
				\[e=0,\quad a^{-1}=-\frac{a}{a+1}\]
			\item[(3)] $G=\bbR\backslash\qty{0},\quad a\circ b=2ab,\quad\qty(G,\circ)$は群をなす.この群の単位元および$a$の逆元を求めよ
				\[e=\frac{1}{2},\quad a^{-1}=\frac{1}{4a}\]
		\end{enumerate}
		これらの例では,より簡単な群と関係づけることができる.以下,準同型,同型を導入する.
		\begin{dfn}
			2つの群$G,G^\prime$に対して,写像$f:G\to G^\prime$が
			\[\redunderline{f\qty(x\,{\color{blue}\circ}\,y)=f\qty(x)\,{\color{blue}\cdot}\,f\qty(y)\quad\qty(x,y\in G)}\]
			を満たすとき,$f$を\redunderline{準同型写像}または\redunderline{準同型(homomorphism)}という.また,$G$から$G^\prime$への準同型写像全体を$\Hom\qty(G,G^\prime)$とかく.
		\end{dfn}
		\begin{remark*}
			$G,G^\prime$として,$\qty(G,\circ),\qty(G^\prime,\cdot)$を意味している.
		\end{remark*}
		\begin{itemize}
			\item $G$上の二項演算$\circ$を$G^\prime$上の二項演算$\cdot$へ移す.
				\[f\qty(x\circ y)=f\qty(x)\cdot f\qty(y)=x^\prime\cdot y^\prime\quad\text{ここで}x^\prime,y^\prime\in G^\prime\]
			\item 群$G$の単位元$e$は,群$G^\prime$の単位元$e^\prime$に移される.
				\begin{align*}
					                & f\qty(e)\cdot f\qty(e)=f\qty(e\circ e)=f\qty(e)                 \\
					\Leftrightarrow & f\qty(e)^{-1}\cdot f\qty(e)\cdot f\qty(e)=f\qty(e)^{-1}f\qty(e) \\
					\Leftrightarrow & \,e^\prime\cdot f\qty(e)=f\qty(e)=e^\prime
				\end{align*}
			\item 群$G$の逆元は,群$G^\prime$の逆元に移される.
				\begin{align*}
					                & f\qty(a)\cdot f\qty(a^{-1})=\qty(a\circ a^{-1})=f\qty(e)=e^\prime \\
					\Leftrightarrow & f\qty(a^{-1})=f\qty(a)^{-1}
				\end{align*}
				\begin{align*}
					\begin{array}{ccc}
						\qty(x,y)\in G^2                        & \stackrel{\circ}{\longmapsto} & x\circ y\in G                      \\
						\downarrow f                            &                               & \downarrow f                       \\
						\qty(f\qty(x),f\qty(y))\in G^{\prime 2} & \longmapsto                   & f\qty(x)\cdot f\qty(y)\in G^\prime
					\end{array}
				\end{align*}
		\end{itemize}
		\begin{dfn}
			準同型$f$が全単射のとき,$f$を\redunderline{同型写像}または\redunderline{同型(isomorphism)}という.同型写像$f:G\to G^\prime$が存在すれば\redunderline{$G$と$G^\prime$は同型}であるといい.$G\simeq G^\prime$または$G\stackrel{\sim}{\longrightarrow}G^\prime$などとかく.
		\end{dfn}
		\begin{example*}
			同型を用いると,先の例(1),(2),(3)は\dots

			(1)の群は,通常の加法に関する$\bbZ$のなす群と同型.$f\qty(a)=a+1$とする.
			\[f\qty(a\circ b)=f\qty(a+b+1)=a+b+2=\qty(a+1)+\qty(b+1)=f\qty(a)+f\qty(b)\]

			(2)と(3)の群は,通常の乗法群としての$\bbR^\times=\bbR\backslash\qty{0}$と同型
			\begin{itemize}
				\item[(2)] $f^\prime\qty(a)=a+1$
				\item[(3)] $f^\prime\qty(a)=2a$
			\end{itemize}
			{\color{red} 準同型,全単射であることは容易に確かめられる.}
		\end{example*}
	\subsubsection*{準同型の例}
		1次分数変換
		\[G=\qty{f\qty(x)=\frac{ax+b}{cx+d}\mid a,b,c,d\in\bbC,ad=bc\neq 0}\]
		$\GL\qty(2,\bbC)$\footnote{$\qty|A|=ad-bc\neq 0$}の元$A=\mqty[
				a & b \\
				c & d
			]$に対して,$G$の元$f_A\qty(x)$を$f_A\qty(x)=\frac{ax+b}{cx+d}$によって定める.
		$\GL\qty(2,\bbC)$の元と$G$の元を対応\footnote{$F:\GL\qty(2,\bbC)\to G$}付けている.
		\[f_{\scriptsize\mqty[
						1 & 0 \\
						0 & 1
					]}\qty(x)=\frac{x+0}{0+1}=x,\quad f_A^{-1}\qty(f_A\qty(x))=x\]
		\[f_A^{-1}\qty(x)=\frac{1}{ad-bc}\frac{dx-b}{-cx+a}=f_{A^{-1}}\qty(x)\]
		ここで$A^{-1}=\frac{1}{ad-bc}\mqty[
				d  & -b \\
				-c & a
			]$であることに注意.

		このとき,任意の$A=\mqty[
				a & b \\
				c & d
			],\quad P=\mqty[
				p & q \\
				r & s
			]\in \GL\qty(2,\bbC)$に対して,
		\begin{align*}
			\begin{array}{ccc}
				\qty(A,P)     & \longmapsto & A\cdot P  \in \GL\qty(2,\bbC) \\
				F\downarrow   &             & F\downarrow                   \\
				\qty(f_A,f_P) & \longmapsto & f_A\circ f_P=f_{AP}
			\end{array}
		\end{align*}
		{\color{red} 対応$F$は$\GL\qty(2,\bbC)\to G$の準同型を与えている.}
		\begin{remark*}
			$G$は写像の合成に関して群.
		\end{remark*}
		$\GL\qty(2,\bbC)$は行列の積に関して群,単位元は$\mqty[1&0\\0&1]$,$A$の逆行列$A^{-1}$が$A$の逆元.
		ただし,この対応$F$は,全単射となっておらず同型ではない.$\Rightarrow$準同型と同型のズレを測るのが「核」である.
		\begin{dfn}
			準同型$f:G\to G^\prime$に対して,
			\[\Ker f=\qty{x\in G\mid f\qty(x)=e^\prime}\]
			を準同型写像$f$の\redunderline{核(kernel)}という.
		\end{dfn}
		\begin{dfn}
			準同型$f:G\to G^\prime$に対して,
			\[\Im f=\qty{f\qty(x)\mid x\in G}\]
			を準同型写像の\redunderline{像(image)}という.
		\end{dfn}
		\begin{example*}
			\begin{align*}
				\begin{array}{ccc}
					f:\GL\qty(2,\bbC)     & \longrightarrow & G=\qty{\phi\qty(z)=\frac{az+b}{cz+d}\mid a,b,c,d\in\bbC,ad=bc\neq 0} \\
					\rotatebox{90}{$\in$} &                 & \rotatebox{90}{$\in$}                                                \\
					\mqty[
					a                     & b                                                                                      \\
					c                     & d
					]                     & \longmapsto     & \phi_A\qty(z)=\frac{az+b}{cz+d}
				\end{array}
			\end{align*}
			は準同型.核
			\begin{align*}
				\Ker f & =\qty{A\in \GL\qty(2,\bbC)\mid\phi_A\qty(z)}     \\
				       & =\qty{\mqty[a                                & 0 \\0&a]\mid a\in\bbC\backslash\qty{0}}
			\end{align*}
			つまり,分子と分母に定数倍の不定性が存在することを示している.
		\end{example*}
		\begin{prop}
			核が単位元のみからなることが,群の準同型が1対1となるための必要十分条件.
		\end{prop}
		\begin{proof}
			(必要条件は明らか)

			十分性を準同型$f:G\to G^\prime$の場合に示す.

			$f\qty(x)=f\qty(y)$とすると,準同型性から
			\[f\qty(x\circ y^{-1})=f\qty(x)\cdot f\qty(y)^{-1}=e\]
			よって核が単位元のみとすると
			\[x\circ y^{-1}=e\Leftrightarrow x=y\]
		\end{proof}
		\begin{prop}
			$f\in\Hom\qty(G,G^\prime),H\leqq G,H^\prime\leqq G^\prime$とする.
			\begin{itemize}
				\item $f\qty(H)\leqq G^\prime$
				\item $f^{-1}\qty(H^\prime)\leqq G$
			\end{itemize}
			\begin{proof}
				(レポート)
			\end{proof}
		\end{prop}
		\begin{cor}
			自明な群$\qty{e^\prime}\subset G^\prime$の逆像である準同期写像$f$の核$\Ker f$は$G$の部分群となる.
		\end{cor}
		\begin{note*}
			ベクトル空間の間の準同型のことを,通常,線形写像と呼ぶ.

			線形写像$\varphi:V\to W$に対して,$\Ker\varphi$は,$V$の部分ベクトル空間.$\Im\varphi$は,$W$の部分ベクトル空間.
		\end{note*}
		一般の準同型$f:G\to G^\prime$は,$G$の元をいくつかずつまとめて,より"粗い"群を作って$G^\prime$に埋め込む.準同型$f$が元の群をどれだけ"粗く"するかは,$f$の像の1点,特に単位元の逆像のみで決まる.
		\begin{dfn}
			$G$を群とする.$G$から$G$への準同型写像を$G$上の\redunderline{自己準同型(endomorphism)}という.$G$上の自己準同型写像の全体$\End\qty(G)$はモノイドをなす.さらに同型写像である時は,$G$上の\redunderline{自己同型(automorphism)}という.$G$上の自己同型全体からなる集合$\Aut\qty(G)$を\redunderline{自己同型群}という.
		\end{dfn}
		\begin{example*}[全単射だが同型でない例]
			\begin{align*}
				G      & :\,\text{位数}n\text{の有限群} \\
				\phi_a & :\,G\to G;\,g\mapsto ag
			\end{align*}
			ここで$a\in G$とする.\footnote{準同型でないことがわかる}
		\end{example*}
	\subsubsection*{自己同型の重要な例}
		\begin{prop}
			群$G$の元$a\in G$を選ぶ.$a\in G$による\redunderline{共役変換}と呼ばれる写像
			\[A_a:G\to G;\,g\mapsto aga^{-1}\]
			で定義する.これは$G$上の自己同型を与える.
		\end{prop}
		\begin{abbr*}
			\subparagraph{準同型性}
				\[A_a\qty(g)A_a\qty(h)=aga^{-1}aha^{-1}=agha^{-1}=A_a\qty(gh)\]
			\subparagraph{全単射}(略)
		\end{abbr*}
\section{群作用と軌道分解}
	\subsection{作用について}
		群$G$の集合$X$への作用について,$G$自身への作用を含めて議論する.群の作用には,左作用と右作用がある.
		\begin{dfn}
			群$G$の集合$X$への\redunderline{左作用(left action)}とは,次の条件を満たす写像$\lambda:G\times X\to X$のこと.
			\begin{itemize}
				\item $\forall x\in X$に対し,
					\[\lambda\qty(e_G,x)=x\]
				\item $\forall g,h\in G,\,\forall x\in X$に対し,
					\[\redunderline{\lambda\qty(gh,x)=\lambda\qty(g,\lambda\qty(h,x))}\]
			\end{itemize}
			このとき,$G$は$\lambda$によって$X$に左から作用(act from left)するという.このような写像$\lambda$が与えられた集合$X$を\redunderline{左$G$集合(left $G$-set)}という.
		\end{dfn}
		\begin{dfn}
			群$G$の集合$X$への\redunderline{右作用(right action)}とは,次の条件を満たす写像$\rho:X\times G\to X$のこと.
			\begin{itemize}
				\item $\forall x\in X$に対し,
					\[\rho\qty(x,e_G)=x\]
				\item $\forall g,h\in G,\,\forall x\in X$に対し,
					\[\redunderline{\rho\qty(x,gh)=\rho\qty(\rho\qty(x,g),h)}\]
			\end{itemize}
			このとき,$G$は$\rho$によって$X$に右から作用(act from right)するという.このような写像$\rho$が与えられた集合$X$を\redunderline{右$G$集合(right $G$-set)}という.
		\end{dfn}
		\begin{remark*}
			明示する必要がない場合,$\lambda\qty(g,x)$を$g\cdot x$や$gx$などと省略することが多い.
		\end{remark*}
		\begin{note*}
			自明な作用:\,全ての$g\in G,\,x\in X$に対し,$\lambda\qty(g,x)=x$とする作用
		\end{note*}
		\begin{note*}
			対称群$\frakS_n$はその定義から,$X=\qty{1,2,\cdots,n}$上に自然に置換として作用している.本講義では通常は左作用としている.
		\end{note*}
		\begin{remark*}
			$\rho\qty(x,g)$を$\qty(x)\rho_g,\,x^g$あるいは$x\cdot g,\,xg$などと省略する.
		\end{remark*}
		\begin{prop}
			群$G$が集合$X$に$\qty(g,x)\mapsto g\cdot s$によって\redunderline{左から作用}しているとき,$\rho\qty(x,g)=x^g=g^{-1}\cdot x$とおくと,これは$G$の右作用となる.逆も同様,$G$の$X$への左作用と$G$の$X$への右作用は1対1となる.
		\end{prop}
		\begin{proof}
			$\qty(G,\circ)$に対し,
			\[\qty(g\circ h,h)\mapsto\qty(g\circ h)\cdot x=g\cdot\qty(h\cdot x)\]
			このとき
			\begin{align*}
				\rho\qty(x,g\circ h) & =\qty(g\circ h)^{-1}\cdot x=\qty(h^{-1}\circ g^{-1})\cdot x=h^{-1}\cdot\qty(g^{-1}\cdot x) \\
				                     & =\rho\qty(\rho\qty(x,g),h)
			\end{align*}
		\end{proof}
		\begin{example*}
			正三角形への作用について,% 書き直す?
			\begin{itemize}
				\item[変換1] $\rho_a$:反時計周りに$\frac{2}{3}\pi$回転\\$\rho_h$:重心を通る垂直軸で折り返し変換
				\item[変換2] $\sigma_{132}=\mqty(1&2&3\\3&1&2)$:頂点番号集合$\qty{1,2,3}$への置換\\$\sigma_{23}=\mqty(1&2&3\\1&3&2)$:頂点番号集合$\qty{1,2,3}$への置換
			\end{itemize}
			\begin{align*}
				\rho_a\qty(\triangle\qty(123))                             & \mapsto\triangle\qty(312)                              \\
				\rho_b\qty(\triangle\qty(123))                             & \mapsto\triangle\qty(132)                              \\
				\sigma_{132}\qty(\triangle\qty(123))                       & \mapsto\triangle\qty(312)                              \\
				\sigma_{23}\qty(\triangle\qty(123))                        & \mapsto\triangle\qty(132)                              \\
				\qty(\rho_a\circ\rho_b)\qty(\triangle\qty(123))            & \mapsto\rho_a\qty(\rho_b\triangle\qty(123))            \\
				                                                           & =\triangle\qty(213)                                    \\
				\qty(\sigma_{132}\circ\sigma_{23})\qty(\triangle\qty(123)) & =\sigma_{132}\qty(\sigma_{23}\triangle\qty(123))       \\
				                                                           & =\sigma_{132}\qty(\triangle\qty(132))                  \\
				                                                           & =\triangle\qty(321)                                    \\
				\qty(\triangle\qty(123))\qty(\rho_a\circ\rho_b)            & =\qty(\qty(\triangle\qty(123))\rho_a)\rho_b            \\
				                                                           & =\qty(\triangle\qty(312))\rho_b=\triangle 321          \\
				\qty(\triangle\qty(123))\qty(\sigma_{132}\circ\sigma_{23}) & =\qty(\qty(\triangle\qty(123))\sigma_{132})\sigma_{23} \\
				                                                           & =\triangle\qty(213)
			\end{align*}
			ここで$x\mapsto f\qty(g^{-1}\circ x)$とすると{\color{red} 左作用}となる.
		\end{example*}
		\begin{prop}[$G$集合$X$上の関数への作用]
			群$G$が集合$X$に\redunderline{左から作用}しているとき,$X$上の複素数値関数全体を$F\qty(X)$と置き,$g\in G$の$F\qty(X)$への作用を,関数$f\in F\qty(X)$を関数$x\mapsto f\qty(g\cdot x)\quad\qty(\forall x\in X)$に写す写像を定める.これは{\color{red} 右作用}になる.これを$G$の$X$への作用から引き起こされる$X$上の関数への作用という.
		\end{prop}
		\begin{example*}
			$\calH=\qty{z\in\bbC\mid\Im z>0}$,左作用$\mu:\SL_2\qty(\bbR)\times\calH\to\calH$
			\[\qty(g=\mqty(a&b\\c&d),z)\mapsto\mu\qty(g,z)=\frac{az+b}{cz+d}\quad\qty(ad-bc=1)\]
			複素上半平面$\calH$の上の正則関数$f\qty(z)$全体の集合$H\qty(\calH)$への作用$f\qty(g\cdot z)$は右作用となる.

			このとき,
			\begin{align*}
				\mu\qty(hg,z) & =\mu\qty(h,\mu\qty(g,z))                                                                          \\
				              & =\mu\qty(\mqty(a^\prime                                                                & b^\prime \\c^\prime&d^\prime),\frac{az+b}{cz+d})\\
				              & =\frac{a^\prime\qty(az+b)/\qty(cz+d)+b^\prime}{c^\prime\qty(az+b)/\qty(cz+d)+d^\prime}
			\end{align*}
			例えば,$g=\mqty(1&1\\0&1),h=\mqty(1&0\\1&1)\in \SL_2\qty(\bbR)$
			\[gh=\mqty(2&1\\1&1),\quad hg=\mqty(1&1\\1&2)\]
			\begin{align*}
				f\qty(z) & \stackrel{g}{\to}F\qty(z)=f\qty(g\cdot z)=f\qty(z+1)                                                                                                    \\
				         & \stackrel{h}{\to}F\qty(h\cdot z)=F\qty(\frac{z}{z+1})=f\qty(\frac{z}{z+1}+1)=f\qty(\frac{2z+1}{z+1})=f\qty(\qty(gh)\cdot z)=f\qty(h\cdot\qty(g\cdot z))
			\end{align*}
		\end{example*}
	\subsection{集合$X$の軌道分解}
		\begin{dfn}
			(二項演算が定義されている)集合$S$の集合$X$への左からの作用とは,次の条件より写像
			\[\lambda:S\times X\to X\]
			のこと.
			\begin{itemize}
				\item 任意の$g,h\in S,\,x\in X$に対し,
					\[\lambda\qty(gh,x)=\lambda\qty(g,\lambda\qty(h,x))\]
			\end{itemize}
		\end{dfn}
		\begin{dfn}
			$\cdot:S\times X\to X$を集合$S$の集合$X$への左からの作用とする.$x\in X$に対し,$x$の$S$-軌道($S$-orbit)を,
			\[S\qty(x)\coloneqq S\cdot x=\qty{s\cdot x\mid s\in S}\subset X\]
			で定義する.
		\end{dfn}
		\begin{prop}
			$\qty(G,\circ)$を群とし,$\cdot :G\times X\to X$を$G$の群としての$X$への左からの作用とする.このとき,\redunderline{二項関係$x\stackrel{G}{\sim}y$を$x\in G\cdot y$}で定義すると,同値関係\footnote{cf.同値関係:集合$X$上の同値関係$\sim$\,$\Leftrightarrow\forall a,b,c\in X$に対し,\begin{itemize}
					\item $a\sim a$
					\item $a\sim b\Rightarrow b\sim a$
					\item $a\sim b$かつ$b\sim c\Rightarrow a\sim c$
				\end{itemize}}になる.
			\begin{align*}
				x\stackrel{G}{\sim}y & \Leftrightarrow\exists g\in G\suchthat x=g\cdot y \\
				                     & \Leftrightarrow G\qty(x)=G\qty(y)
			\end{align*}
		\end{prop}
		ここで$x\stackrel{G}{\sim}y$は同値関係なので同値類に関する類別を考えることができる.これを\redunderline{軌道分解}という.
		\begin{dfn}
			$s\stackrel{G}{\sim}y$に関して$X$を同値類に分割したものを,$X$の$G$による\redunderline{軌道分解}といい$X/\sim G$で表す.
			\[X/\sim G=\qty{G\qty(t)\mid t\in\Lambda},\suchthat X=\bigsqcup_{t\in\Lambda}G\qty(t)\]
			ここで$\Lambda$は完全代表系,集合$X$の群$G$による軌道分解を$G\backslash X$で表すことも多い.また,軌道が唯一つになるとき,作用が\redunderline{推移的}(transitive)であるという.
		\end{dfn}
		\begin{note*}
			$G$の$X$への作用が推移的\\
			$\Leftrightarrow\forall x,y\in X$に対し,$\exists g\in G\suchthat x=g\cdot y$\\
			$\Leftrightarrow\forall x_0\in X$に対し,$G\cdot x_0=G\qty(x_0)=X$
		\end{note*}
		\begin{example*}[軌道分解の例] %TODO:図を書く
			4次の巡回群$C_4$の正六面体に作用
			\begin{align*}
				C_4\cdot 1 & =\qty{1,2,3,4} \\
				C_4\cdot 5 & =\qty{5,6,7,8}
			\end{align*}
			\begin{figure}[h]
				\centering
				\tdplotsetmaincoords{60}{30}
				\begin{tikzpicture}[tdplot_main_coords]
					\draw (-1,-1,1)node[left]{1};
					\draw (1,-1,1)node[right]{2};
					\draw (1,1,1)node[right]{3};
					\draw (-1,1,1)node[above]{4};
					\draw (-1,-1,-1)node[left]{5};
					\draw (1,-1,-1)node[right]{6};
					\draw (1,1,-1)node[right]{7};
					\draw (-1,1,-1)node[left]{8};
					\draw (-1,-1,1)--(1,-1,1)--(1,1,1)--(-1,1,1)--(-1,-1,1);
					\draw (-1,-1,1)--(-1,-1,-1)--(1,-1,-1)--(1,-1,1);
					\draw (1,1,1)--(1,1,-1)--(1,-1,-1);
					\draw[dashed] (-1,1,1)--(-1,1,-1)--(-1,-1,-1);
					\draw[dashed] (1,1,-1)--(-1,1,-1);
					\fill[red] (0,0,1)circle(0.05);
					\fill[red] (0,0,-1)circle(0.05);
					\draw[->][red] (0,0,-2.5)--(0,0,3);
					\draw[->][red] (0.25,0,2.5)arc(0:270:0.25);
				\end{tikzpicture}
				\caption{正六面体}
			\end{figure}
		\end{example*}
		\begin{dfn}[固定部分群]
			群$G$が集合$X$に作用しているとき,$X$の1点$x$を動かさない$g\in G$の全体は,$G$の部分群となる.これを$x$の\redunderline{固定部分群}(stabilizer of $x$)といい,$G_x$とかく.
			\[G_x=\qty{g\in G\mid g\cdot x}\qty(\leqq G)\]
		\end{dfn}
		\begin{example*}
			正六面体群$P\qty(6)$において頂点$A_1$を固定する部分群は位数3の巡回部分群.
			\[P\qty(6)_{A_1}=\qty{\text{恒等変換},\text{立方体の重心と頂点}A_1\text{を通る対角線周りの}\frac{2}{3}\pi\text{回転},\frac{4}{3}\pi\text{回転}}\]
		\end{example*}
		\begin{itemize}
			\item 集合$X$への群$G$の作用が推移的である時,集合$X$と集合$G$は以下の意味で"ほぼ同じ"とみなすことができる.
		\end{itemize}
		\begin{prop}
			$G$を群,$X$を$G$-集合.$x_0\in X$に対し,
			\[f:G\to X;\,g\mapsto g\cdot x_0\]
			を定める.このとき,
			\[G\text{の$X$への作用が推移的}\Leftrightarrow f\text{が全射},\,f\qty(G)=X\]
		\end{prop}
		\begin{remark*}
			このとき,$f$は単射であるとは限らない.しかし,群$G$を固定部分群で"割る"ことで全単射を作ることができる.(cf.準同型定理など\dots)

			「群$G$を部分群で割る」→$G$の同値類のひとつである\blueunderline{剰余類}の導入
		\end{remark*}
	\subsection{群$G$の剰余類分解}
		特に,部分群の$G$自身への作用による軌道分解
		\begin{itemize}
			\item 部分群$H$の群$G$への右作用乗法移動
				\[\rho:G\times H\to G;\,\qty(g,h)\mapsto gh\]
			\item 部分群$H$の群$G$への左作用乗法移動
				\[\lambda:H\times G\to G;\,\qty(h,g)\mapsto hg\]
		\end{itemize}
		$\Rightarrow$対応する軌道分解によって,群$G$は部分群による同値類に分解(類別)できる.

		以下$H\leqq G$とする.
		\paragraph{部分群$H$の群$G$への\redunderline{右からの乗法移動}による$g\in G$の軌道}
			\[H\qty(g)=\qty{\rho\qty(g,h)\mid h\in H}=\qty{g\cdot h\mid h\in H}=gH\]
			による軌道分解$G=\bigsqcup_{a\in\Lambda}aH$から誘導される同値関係,同値類,剰余(商)集合が得られる.

			\begin{prop}[$H$軌道から誘導される$G$上の同値関係]
				$H\leqq G$とする.$a,b\in G$に対する関係$a\sim b$
				\[a\sim b\Leftrightarrow a^{-1}b\in H\]
				は同値関係.$\stackrel{\text{L}}{\sim}$と表すこともある.
			\end{prop}
			\begin{proof}
				$a\sim a$ : $a^{-1}a=e\in H.$

				$a\sim b\Leftrightarrow b\sim a$ : $a^{-1}b\in H\Leftrightarrow\qty(a^{-1}b)^{-1}=b^{-1}a\in H$

				$a\sim b$かつ$b\sim c\Rightarrow a\sim c$ : $a^{-1}b,b^{-1}c\in H$とすると,
				\[a^{-1}c=\qty(a^{-1}b)\cdot \qty(b^{-1}c)\in H\]
			\end{proof}
			\begin{note*}
				同値関係$\stackrel{\text{L}}{\sim}$による$a$の同値類$aH$
				\[C_H\qty(a)=\qty{b\in G\mid a\stackrel{\text{L}}{\sim}b}=\qty{b\in G\mid a^{-1}b\in H}\]
				と表すこともある.
			\end{note*}
			このとき,$a\sim b$ならば,$\exists h\in H\suchthat a^{-1}b=h\Leftrightarrow b=ah$
			\begin{dfn}
				$H\leqq G$.$G$の部分集合$gH=\qty{gh\mid h\in H}$を$g$の\redunderline{左剰余類}(left coset)とよび,同値関係$\stackrel{\text{L}}{\sim}$による$g$の同値類を表す.
			\end{dfn}
			\begin{dfn}[剰余集合・商集合]
				(左)剰余類$aH$全体のなす集合\redunderline{(左)剰余集合}または\redunderline{商集合}と呼ぶ.群$G$の部分群$H$による左剰余集合を$G/H$で表す.

				また,各類から1つずつ代表元を取ってきた代表元全体の集合$\Lambda$を$G$の$H$に関する\redunderline{左完全代表系}という.
			\end{dfn}
			このとき,相異なる左剰余類の個数を$H$の$G$による\redunderline{指数}(index)といい,$\qty(G:H)$で表す.指数が$\infty$となる場合もある.定義より
			\[\qty(G:H)=\qty|G/H|\]
			である.

			部分群$H$による群$G$の左剰余類への分解(coset decomposition)(左類別ともいう).
			\[G=\bigsqcup_{a\in\Lambda}aH\]
			このとき,左完全代表形$\Lambda=\qty{a^\prime,b^\prime,\cdots}$とすると
			\begin{align*}
				G/H           & =\qty{a^\prime H,b^\prime H,\cdots } \\
				\qty|\Lambda| & =\qty|G/H|=\qty(G:H)
			\end{align*}
		\paragraph{部分群の群$G$への\redunderline{左からの乗法移動}による$g\in G$の$H$軌道}
			\[H\qty(g)=\qty{\lambda\qty(h,g)\mid h\in H}=\qty{h\cdot g\mid h\in H}=Hg\]
			による軌道分解(右類別ともいう)
			\[G=\bigsqcup_{a\in\Lambda^\prime}Ha\]
			から誘導される同値関係・同値類・剰余(商)集合が得られる.
			\begin{prop}[左乗法移動から得られる$H$軌道が誘導する$G$上の同値関係]
				$H\leqq G$とする.$a,b\in G$に対する関係$a\sim b$
				\[a\sim b\Leftrightarrow ba^{-1}\in H\]
				は同値関係となる.以降$\stackrel{\text{R}}{\sim}$と表す.
			\end{prop}
			この同値関係$\stackrel{\text{R}}{\sim}$による剰余類
			\[C^\prime_H\qty(a)=\qty{x\in G\mid x\stackrel{\text{R}}{\sim}a}\]
			を$G$の$H$による\redunderline{右剰余類}という.$Ha$と表すことも多い
			\begin{example*}
				3次対称群$\frakS_3$に対して,部分群$H_1\qty(\simeq C_2), H_2\qty(\simeq C_3)$を次で定める.
				\begin{align*}
					H_1      & =\qty{e,\qty(1\ 2)},                                                       \\
					H_2      & =\qty{e, \qty(1\ 2\ 3), \qty(1\ 3\ 2)},                                    \\
					\frakS_3 & =\qty{e, \qty(1\ 2), \qty(1\ 3), \qty(2\ 3), \qty(1\ 2\ 3), \qty(1\ 3\ 2)}
				\end{align*}
				において,
				\begin{align*}
					eH_1          & =\qty{e, \qty(1\ 2)},                                \\
					\qty(1\ 2)H_1 & =\qty{\qty(1\ 2)e=\qty(1\ 2),\qty(1\ 2)\qty(1\ 2)=e}
				\end{align*}
				よって
				\[eH_1=\qty(1\ 2)H_1\]
				左剰余集合$\frakS_3/H_1=\qty{eH_1,\qty(1\ 3)H_1}$
				\begin{align*}
					\qty(1\ 3)H_1         & =\qty{\qty(1\ 3)e=\qty(1\ 3),\qty(1\ 3)\qty(1\ 2)=\qty(1\ 2\ 3)}=\qty(1\ 2\ 3)H_1            \\
					\qty(2\ 3)H_1         & =\qty{\qty(2\ 3)e=\qty(2\ 3),\qty(2\ 3)\qty(1\ 2)=\qty(1\ 3\ 2)}=\qty(1\ 3\ 2)H_1            \\
					\frakS_3/H_1          & =\qty{e,\qty(1\ 2)}\sqcup\qty{\qty(1\ 3), \qty(1\ 2\ 3)}\sqcup\qty{\qty(2\ 3),\qty(1\ 3\ 2)} \\
					H_1\backslash\frakS_3 & =\qty{H_1e=H_1\qty(1\ 2),H_1\qty(1\ 3)=H_1\qty(1\ 3\ 2),H_1\qty(2\ 3)=H_1\qty(1\ 2\ 3)}      \\
					H_1\qty(1\ 3)         & =\qty{e\qty(1\ 3)=\qty(1\ 3),\qty(1\ 2)\qty(1\ 3)}=\qty(1\ 3\ 2)=H_1\qty(1\ 3\ 2)            \\
					                      & =\qty{e, \qty(1\ 2)}\sqcup\qty{\qty(1\ 3),\qty(1\ 3\ 2)}\sqcup\qty{\qty(2\ 3),\qty(1\ 2\ 3)}
				\end{align*}
				{\color{red} ここで
				\[\frakS_3/H_1\neq H_1\backslash\frakS_3\]
				である.}

				\underline{$H_2$による左剰余類の集合}
				\begin{align*}
					\frakS_3/H_2          & =\qty{eH_2,\qty(1\ 2)H_2},                                                                             \\
					eH_2                  & =\qty{e,\qty(1\ 2\ 3),\qty(1\ 3\ 2)},                                                                  \\
					\qty(1\ 2)H_2         & =\qty{\qty(1\ 2),\qty(1\ 2)\qty(1\ 2\ 3)=\qty(2\ 3),\qty(1\ 2)\qty(1\ 3\ 2)=\qty(1\ 3)}                \\
					H_2\backslash\frakS_3 & =\qty{H_2e,H_2\qty(1\ 2)}                                                                              \\
					H_2e                  & =H_2,                                                                                                  \\
					H_2\qty(1\ 2)         & =\qty{e\qty(1\ 2),\qty(1\ 2\ 3)\qty(1\ 2)=\qty(1\ 3),\qty(1\ 3\ 2)\qty(1\ 2)=\qty(2\ 3)},              \\
					\frakS_3/H_2          & =\qty{e,\qty(1\ 2\ 3),\qty(1\ 3\ 2)}\sqcup\qty{\qty(1\ 2),\qty(1\ 3),\qty(2\ 3)}=H_2\backslash\frakS_3
				\end{align*}
				{\color{red} ここで
				\[\frakS_3/H_2= H_2\backslash\frakS_3\]
				である.}
			\end{example*}
\section{正規部分群と剰余群}
	\begin{prop}
		$H$を$G$の部分群とする.$G$の$H$による左類別と右類別が類別として一致するための必要十分条件は任意の$a\in G$に対して
		\[aHa^{-1}=H\]
		が成立すること.
	\end{prop}
	\begin{dfn}[正規部分群]
		$N\leqq G, G$の任意の元$g$に対して,$gNg^{-1}=N$とする.このとき,$N$を$G$の\redunderline{正規部分群}(normal subgroup)という.記号として,$N\triangleleft G$または$G\triangleright N$と書くことにする.
	\end{dfn}
	\begin{prop}
		$H\leqq G$,任意の$g\in G$に対して$gHg^{-1}\subset H$ならば,$H\triangleleft G$.
	\end{prop}
	\begin{proof}
		$\forall g\in G$に対し,$gHg^{-1}\subset H$ならば,
		\[H\subset g^{-1}Hg=g^{-1}H\qty(g^{-1})^{-1}\]
		であり,$g^{-1}$も$G$の任意の元を表すので,
		\[gHg^{-1}\subset H\Rightarrow H\subset gHg^{-1}\]
		である.よって
		\[gHg^{-1}=H\]
	\end{proof}
	\begin{screen} % TODO: この枠は赤色
		$N$を群$G$の正規部分群とすると,$N$による剰余(商)集合に群構造を導入することができる.
	\end{screen}
	\begin{prop}
		$N\triangleleft G$.このとき,剰余集合$G/N$は自然に群をなす.すなわち,$G/N$上の群演算を
		\[xN\cdot yN=\qty(xy)N\]
		によって定義することができる.
	\end{prop}
	\begin{proof}
		well-defined であることを確かめる(\redunderline{演算の定義が剰余類の代表元の選び方によらない}ことを示す必要がある).つまり,
		\[xN=x^\prime N,yN=y^\prime N\Rightarrow xyN=x^\prime y^\prime N\]
		を示す必要がある.このとき,$x=x^\prime h, y=y^\prime h^\prime$を満たす$h,h^\prime\in N$が存在する.よって
		\[xy=x^\prime hy^\prime h^\prime=x^\prime y^\prime\qty({y^{\prime}}^{-1}hy^\prime)h^\prime\in x^\prime y^\prime N\]
		ここで${y^\prime}^{-1}hy^\prime\in {y^\prime}^{-1}Ny^\prime=N$.

		(その他の群としての性質についても省略)
	\end{proof}
	\begin{dfn}
		$N\triangleleft G$.このとき,剰余集合$G/N$を剰余群(residue class group)または商群(quotient group)という.
	\end{dfn}
	\begin{remark*}
		$N\triangleleft G$とする.剰余群$G/N$の単位元は$C_N\qty(e)=eN$.剰余群$G/N$の元$C_N\qty(a)=aN$の逆元は$C_N\qty(a^{-1})=a^{-1}N$
	\end{remark*}
	\begin{dfn}[単純群]
		正規部分群が(単位群と自分自身以外に)1つもない群を単純群という.\\
		cf. 有限単純群の分類(巡回,交代,リー型,散在)モンスター群の位数$\approx 8\times 10^{53}\gg 6\times 10^{23}$(アボガドロ数).
	\end{dfn}
	\begin{example*}
		$n$次交代群$\frakA_n$は$n$次対称群$\frakS_n$の正規部分群となる.
	\end{example*}
	\begin{dfn}
		群$G$の中心$Z\qty(G):G$の全ての元と可換な元全体の集合.
	\end{dfn}
	\begin{remark*}
		$Z\qty(G)$は,$G$の正規部分群となる.このとき,射影一般線形群$\PGL_2\qty(\bbR)\coloneqq \GL\qty(2;\bbR)/Z\qty(\GL\qty(2;\bbR))$,射影特殊線形群$\PSL_2\qty(\bbR)\coloneqq \SL\qty(2;\bbR)/Z\qty(\SL\qty(2;\bbR))$
	\end{remark*}
	\subsection{ラグランジュの定理}
		\begin{prop}[左剰余類に関する定理]
			$H\leqq G,a,b\in G$に対して
			\begin{itemize}
				\item $a\sim b\Leftrightarrow aH=bH\Leftrightarrow aH\cap bH=\emptyset$
				\item $a\not\sim b\Leftrightarrow aH\cap bH=\emptyset$
				\item $a\in G$に対して,写像$\varphi_a:H\to aH;h\mapsto ah$
			\end{itemize}
			このとき,$\varphi_a$は全単射である(cf.組み替え定理).
		\end{prop}
		このとき$g\in G$において,$gH$や$Hg$などの部分集合は,$H$の一対一写像による像なので,元の個数は一定で,すべての$H$の位数と一致する.

		さらに,$G=\bigsqcup_{g\in\Lambda}gH$より,$H$の位数を$H$の$G$における剰余類の総数$\qty(G:H)$を用いて$G$の位数は次で表せる.
		\begin{thm}[ラグランジュの定理]
			有限群$G$の部分群を$H$とすれば,
			\[\qty|G|=\qty(G:H)\qty|H|\]
			が成り立つ.よって,部分群$H$の位数$\qty|H|$,指数$\qty(G:H)$はどちらも$\qty|G|$の約数である.
		\end{thm}
		\begin{cor}
			有限群$G$の元を$a$とすれば,$a$の位数は$\qty|G|$の約数である.従って$a^\qty|G|=e$.(逆は成り立たない)
		\end{cor}
		\begin{proof}
			位数$m$の元$g$は群$G$の中に$m$次の巡回群を生成する.
		\end{proof}
		\begin{cor}
			有限群$G$において,位数$\qty|G|$が素数ならば,$G$は巡回群である.
		\end{cor}
	\subsection{群$G$への共役類分解}
		\begin{screen}%TODO: 赤線
			$G$自身への(左からの)共役作用について考える.
			\[G\times G\to G;\qty(g,x)\mapsto gxg^{-1}\]
		\end{screen}
		群$G$による$a\in G$への共役作用による軌道$\qty[a]$
		\[\qty[a]\coloneqq G\qty(a)=\qty{gag^{-1}\mid g\in G}\]
		とその軌道分解
		\[G=\bigsqcup_{a\in\Lambda}\qty[a]\]
		から誘導される同値関係と同値類
		\begin{dfn}[共役な元,共役類]
			群$G$上の同値関係である共役$\stackrel{\text{conj}}{\sim}$
			\[x\stackrel{\text{conj}}{\sim}y\Leftrightarrow\exists g\in G\suchthat gxg^{-1}=y\]
			この同値関係$\stackrel{\text{conj}}{\sim}$による$a\in G$の同値類$\qty[a]$を\redunderline{共役類(conjugacy class)}という.$a$の共役類$\qty[a]=\qty{gag^{-1}\mid g\in G}$.
		\end{dfn}
		\begin{dfn}[共役変換]
			$g\in G$による共役変換
			\[A_g:G\to G;a\mapsto gag^{-1}\]
			$\Rightarrow$群自己同型写像.
		\end{dfn}
		\begin{dfn}[共役.共役部分群]
			部分集合$H_1,H_2\subset G$に対して$H_1=gH_2g^{-1}$なる元が存在するとき,$H_1$と$H_2$は共役であるという.特に,$H_1$が群$G$の部分群のとき,$H_2$は$H_1$の共役な部分群であるという.
		\end{dfn}
		\paragraph{共役類の性質}
			\begin{itemize}
				\item 単位元は単独で類をなす.
				\item 可換群は,各元が単独で共役類.
			\end{itemize}
			\begin{example*}[$\frakS_3$の共役類]
				\[\frakS_3=\qty{e,c,c^{-1},\sigma_1,\sigma_2,\sigma_3}\]
				ここで$c=\qty{1\ 2\ 3},\sigma_1=\qty(1\ 2),\sigma_2=\qty(1\ 3),\sigma_3=\qty(2\ 3)$
				\begin{align*}
					\qty[e]        & =\qty{e}                          \\
					\qty[c]        & =\qty{c,c^{-1}}                   \\
					\qty[\sigma_1] & =\qty{\sigma_1,\sigma_2,\sigma_3}
				\end{align*}
				\begin{table}[h]
					\centering
					\begin{tabular}{c|cccccc}
						$g\backslash a$  & $e$  & $c$       & $c^{-1}$  & $\sigma$    & $\sigma$    & $\sigma$    \\\hline
						$e$              & $e$  & $c$       & $c^{-1}$  & $\sigma_1$  & $\sigma_2$  & $\sigma_3$  \\
						$c$              & $e$  & $c$       & $c^{-1}$  & $\sigma_2$  & $\sigma_3$  & $\sigma_1$  \\
						$c^{-1}$         & $e$  & $c$       & $c^{-1}$  & $\sigma_3$  & $\sigma_1$  & $\sigma_2$  \\
						$\sigma_1$       & $e$  & $c^{-1}$  & $c$       & $\sigma_1$  & $\sigma_3$  & $\sigma_2$  \\
						$\sigma_2$       & $e$  & $c^{-1}$  & $c$       & $\sigma_3$  & $\sigma_2$  & $\sigma_1$  \\
						$\sigma_3$       & $e$  & $c^{-1}$  & $c$       & $\sigma_2$  & $\sigma_1$  & $\sigma_3$
					\end{tabular}
				\end{table}
			\end{example*}
\section{準同型定理}
	\redunderline{準同型写像}は,1対1の写像とも,上への写像とも限らない.しかし,その中から\redunderline{同型写像}を取り出すことができる.
	\begin{prop}
		$G$を群.$N\triangleleft G$とする.このとき,標準射影(自然な全射)
		\[\kappa:G\to G/N;a\mapsto\kappa\qty(a)=C_N\qty(a)=aN\]
		は準同型写像となる.
	\end{prop}
	\begin{proof}
		$a,b\in G$とする.このとき,$\kappa\qty(ab)=\qty(ab)N=aN\cdot bN=\kappa\qty(a)\cdot\kappa\qty(b).$
	\end{proof}
	\begin{prop}
		準同型写像の核は,正規部分群.
	\end{prop}
	\begin{proof}
		$f\in\Hom\qty(G,G^\prime)$とする.
		\subparagraph{部分群であること}
			$\Ker f$には少なくとも$G$の単位元が属している.$a,b\in \Ker f$に対して,$f\qty(ab^{-1})=f\qty(a)f\qty(b)^{-1}=e_{G^\prime}\cdot e_{G^\prime}^{-1}=e_{G^\prime}\Rightarrow ab^{-1}\in\Ker f$が成立.よって$\Ker f$は$G$の部分群.
		\subparagraph{正規であること}
			任意の$g\in G$に対して$a\in\Ker f$ならば,
			\[f\qty(gag^{-1})=f\qty(g)\cdot f\qty(a)\cdot f\qty(g)^{-1}=f\qty(g)\cdot e_{G^\prime}\cdot f\qty(g)^{-1}=e_{G^\prime}\]
			より$gag^{-1}\in\Ker f$が成り立つ.
	\end{proof}
	\begin{thm}[準同型定理]
		$f\in\Hom\qty(G,G^\prime)$,$f$の像$\Im f$のなす$G^\prime$の部分群について,
		\[\bar{f}:G/\Ker f\to\Im f;g\Ker f\mapsto f\qty(g)\]
		による,群の同型\redunderline{$G/\Ker f\simeq\Im\qty(f)$}が成り立つ\footnote{$\kappa$:準同型,$\Ker f$:$G$の正規部分群}.この$\bar{f}$を\redunderline{$f$から誘導される(引き起こされる)同型写像}という.
	\end{thm}
	\begin{align*}
		\begin{array}{ccccc}
			G &                & \stackrel{f}{\longrightarrow} &                                                           & \Im f\subset G^\prime \\
			  & \kappa\searrow &                               & \rotatebox{45}{$\stackrel{\sim}{\longrightarrow}$}\bar{f} &                       \\
			  &                & G/\Ker f                      &                                                           &
		\end{array}
	\end{align*}
	\begin{align*}
		\begin{array}{ccc}
			\bar{f}:G/\Ker f      & \longrightarrow & \Im f                 \\
			\rotatebox{90}{$\in$} &                 & \rotatebox{90}{$\in$} \\
			g\Ker f               & \longmapsto     & f\qty(g)
		\end{array}
	\end{align*}
	\begin{proof}
		\subparagraph{写像の定義が代表元によらないこと}$g^\prime\in g\Ker f$より,
			\[g^\prime=gn\quad\qty(n\in\Ker f)\]
			\[f\qty(g^\prime)=f\qty(gn)=f\qty(g)\cdot f\qty(n)=f\qty(g)\]
		\subparagraph{準同型}
			\begin{align*}
				\bar{f}\qty(g\Ker f\cdot g^\prime\Ker f) & =\bar{f}\qty(\qty(gg^\prime)\Ker f)                   \\
				                                         & =f\qty(gg^\prime)=f\qty(g)\cdot f\qty(g^\prime)       \\
				                                         & =\bar{f}\qty(g\Ker f)\cdot\bar{f}\qty(g^\prime\Ker f)
			\end{align*}
		\subparagraph{単射} $\bar{f}\qty(g\Ker f)=\bar{f}\qty(g^\prime\Ker f)$つまり,$f\qty(g)=f\qty(g^\prime)$とすると,$f\qty(g^{-1}\cdot g^\prime)=f\qty(g)^{-1}\cdot f\qty(g^\prime)=e_{G^\prime}$となり,$g^{-1}\cdot g^\prime\in\Ker f$である.よって$g^\prime\in g\Ker f$から,$g\Ker f=g^\prime\Ker f$が得られる.
		\subparagraph{全射} 任意の$\tilde{g}\in\Im f$から$\bar{f}\qty(g\Ker f)=\tilde{g}$を満たす$g\Ker f\in G/\Ker f$が存在することを示す.$\Im f$の定義より,$\forall\tilde{g}\in\Im f$に対して$f\qty(g)=\tilde{g}$を満たす$g\in G$が存在する.この$g$の属する剰余集合$g\Ker f$を取れば,$\bar{f}\qty(g\Ker f)=f\qty(g)=\tilde{g}$である.
	\end{proof}
	\begin{cor}
		特に$G$が有限群のとき,
		\[\qty|\Im f|=\qty|G|/\qty|\Ker f|\]
		が成り立つ.
	\end{cor}
	\begin{note*}
		線形代数の線形写像に関する次元定理に相当する.
	\end{note*}
	\begin{example*}[対称群$\frakS_n$と交代群$\frakA_n$]
		$\frakS_n$から積を演算とする$\bbR^\times\coloneqq\bbR\backslash\qty{0}$への写像
		\[\sgn:\frakS_n\to\bbR^\times;\sigma\mapsto\sgn\sigma=\begin{cases}
				+1 & \sigma:\text{偶置換} \\
				-1 & \text{それ以外}
			\end{cases}\]
		は準同型写像であり,$\Ker\sgn=\frakA_n,\Im\sgn=\qty{+1,-1}$である.
		\[\bar{f}:\frakS_n/\frakA_n\to\qty{\pm 1};\sigma\frakA_n\mapsto\sgn\sigma\]
		例えば,$\sigma_1=\qty(1, 2)$とすると
		\[\frakS_n/\frakA_n=\qty{e\frakA_n,\sigma_1\frakA_n}\]
		となり,
		\[\bar{f}\qty(e\frakA_n)=1,\bar{f}\qty(\sigma_1\frakA_n)=-1\]
		これは,$\qty{\pm 1}$への同型写像.
	\end{example*}
	\begin{example*}
		$\GL\qty(n;\bbC)$から$\bbC^\times\coloneqq\bbC\backslash\qty{0}$への写像$\det:A\mapsto\det A$は準同型であり,
		\begin{align*}
			\Ker\det & =\SL\qty(n;\bbC)=\qty{g\in \GL\qty(n,\bbC)\mid\det g=1} \\
			\Im\det  & =\bbC^\times=\bbC\backslash\qty{0}
		\end{align*}
		である.
		\[\GL\qty(n;\bbC)/\SL\qty(n;\bbC)\simeq\bbC^\times\]
	\end{example*}
\section{群の乗法作用の軌道について}
	$t$は$G$集合$T$に属する.
	\begin{align*}
		G_t      & = \qty{g\in G\mid gt = t}        & : & t\text{の固定部分群}      \\
		G\qty(t) & = \qty{g\cdot t\in T\mid g\in G} & : & t\text{の}G\text{軌道}
	\end{align*}
	\begin{prop}
		$T$を$G$集合とする.
		\begin{enumerate}
			\item $a\in G,t\in T$に対し,$G_{a\cdot t}=a\cdot G_t\cdot a^{-1}$
			\item $a\in G,t\in T$に対し,$\qty|G_t|=\qty|G_{a\cdot t}|$
		\end{enumerate}
	\end{prop}
	\begin{proof}
		\begin{enumerate}
			\item $g\in G_{a\cdot t}$とすると$gat=at\Leftrightarrow a^{-1}gat=t\Leftrightarrow a^{-1}ga\in G_t\Leftrightarrow g\in aG_ta^{-1}.$
			\item $\qty|G_{a\cdot t}|=\qty|aG_ta^{-1}|=\qty|G_t|$
		\end{enumerate}
	\end{proof}
	\begin{thm}[軌道構造定理]
		$G$集合$T$において$t_0\in T$を1つ選んで固定.$G_{t_0}$と$G\qty(t_0)$に対して
		\[\bar{f}:G/G_{t_0}\to G\qty(t_0);aG_{t_0}\mapsto a\cdot t_0\]
		このとき,$\bar{f}$は全単射である.すなわち,固定部分群$G_{t_0}$に関する$G$の左剰余類と軌道$G\qty(t_0)$の点とは一対一に対応する.
	\end{thm}
	\begin{proof}
		\subparagraph{well-definedであること}
			$\forall b\in aG_{t_0}$とする.このとき,$\exists g\in G_{t_0}\suchthat b=ag\Rightarrow b\cdot t_0=\qty(ag)\cdot t_0=a\cdot\qty(g\cdot t_0)=a\cdot t_0$
		\subparagraph{全射}
			$G\qty(t_0)$の元$a\cdot t_0\,\qty(a\in G)$に対して$G/G_{t_0}$の元$aG_{t_0}$を取ればよい.
		\subparagraph{単射}
			$G/G_{t_0}$の元$aG_{t_0},bG_{t_0}$に対し,
			\begin{align*}
				\bar{f}\qty(aG_{t_0})=\bar{f}\qty(bG_{t_0}) & \Leftrightarrow a\cdot t_0=b\cdot t_0                  \\
				                                            & \Leftrightarrow \qty(a^{-1}b)t_0=t_0                   \\
				                                            & \Leftrightarrow a^{-1}b\in G_{t_0}                     \\
				                                            & \Leftrightarrow\exists g\in G_{t_0}\suchthat g=a^{-1}b
			\end{align*}
			このとき$bG_{t_0}=agG_{t_0}=aG_{t_0}$が成り立つ.
	\end{proof}
	\begin{cor}\label{cor:6-3}
		$G$を有限群,$T$を$G$集合とする.$t\in T$に対して,$t$の軌道$G\qty(t)$内の点の個数
		\[\qty|G\qty(t)|=\qty(G;G_t)=\frac{\qty|G|}{\qty|G_t|}\]
	\end{cor}
	次に軌道がいくつあるのかを数えるのが次の定理(数え上げ問題).
	\begin{thm}[コーシー・フロベニウス]
		$G$集合$T$が$m$個の軌道に分解されるとする.
		\[T=T_1\sqcup T_2\sqcup\cdots\sqcup T_m,\quad T_j\cap T_k=\emptyset\,\qty(j\neq k)\]
		$g\in G$による固定元の集合を$T^g$で表す.
		\[T^g=\qty{t\in T\mid g\cdot t=t}\]
		このとき,軌道の個数$m$は
		\[m=\frac{1}{\qty|G|}\sum_{g\in G}\qty|T^g|\]
		と表される.すなわち,軌道の個数は,固定元の個数の平均値.
	\end{thm}
	\begin{proof}
		積集合$\qty(G,T)=\qty{\qty(g,t)\mid g\in G,t\in T}$を考える.$\qty(G,T)$の部分集合$S$を$S=\qty{\qty(g,t)\mid g\in G,t\in T,\redunderline{gt=t}}$で定義するとき,$S$は次の2通りに分解することができる.
		\begin{align*}
			S & =\bigcup_{g\in G}\qty{\qty(g,t)\mid t\in T^g} &  & \qty(g\in G\text{ごとに固定元をまとめる})   \\
			  & =\bigcup_{t\in T}\qty{\qty(g,t)\mid g\in G_t} &  & \qty(t\in T\text{ごとに固定部分群をまとめる})
		\end{align*}
		このとき,
		\[\qty|S|=\sum_{g\in G}\qty|T^g|=\sum_{t\in T}\qty|G_t|\]
		ここで右辺の和を軌道ごとの和に書き直すと
		\[\sum_{t\in T}\qty|G_t|=\sum_{j=1}^m\sum_{t\in T_j}\qty|G_t|\]
		となる.同じ軌道に属する$t$に対して$\qty|G_t|$の値は一定であり,$T_j$の代表元を$t_j$とすると$T_j=G\qty(t_j)$である.さらに系\ref{cor:6-3}の結果を用いると
		\[\sum_{t\in T_j}\qty|G_t|=\qty|G\qty(t_j)|\cdot\qty|G_{t_j}|=\frac{\qty|G|}{\qty|G_{t_j}|}\qty|G_{t_j}|=\qty|G|\]
		以上より
		\[\sum_{g\in G}\qty|T^g|=\sum_{t\in T}\qty|G_t|=m\qty|G|\]
	\end{proof}
\end{document}
