\documentclass[main]{subfiles}

\usepackage{subfiles}

\begin{document}
\section{準備}
	集合$X,Y,Z$に対して
	\begin{itemize}
		\item $Y\backslash X$:$X,Y\subset Z$に対して,$X$に属さない$Y$の元のなす集合\[Y\backslash X=\qty{y\mid y\in Y\text{\,かつ\,}y\not\in X}\]
		\item $X\sqcup Y$:集合$X$と$Y$の直和(direct sum)あるいは分割和(disjoint sum)を表し,互いに交わらない二つの集合$X$と$Y$の和集合\[A\sqcup B\iff A\cup B\text{\quad for\quad}A\cap B=\emptyset\]
		\item (有限)集合$X$の元の個数を$\sharp X$\[\sharp\qty{a,b,c}=3\]
	\end{itemize}

	\underline{写像} 集合$X$から集合$Y$への写像$f$とは,$X$の各元$x$に対して$Y$の元$y=f\qty(x)$を対応させる規則
	\[f:X\to Y; x\mapsto y=f\qty(x)\]
	\begin{itemize}
		\item $f$による$A\subset X$の像(image)\[f\qty(A)=\qty{f\qty(a)\mid a\in A}\subset Y\]
		\item $f$による$B\subset Y$の逆像あるいは$f$による引き渡し\[f^{-1}\qty(B)=\qty{a\in X\mid f\qty(a)\in B}\]
		\item $f:X\to Y, g:Y\to Z$において,合成$g\circ f$
			\begin{align*}
				\begin{array}{ccccc}
					g\circ f: & X                     & \longrightarrow & Z                     &                                  \\
					          & \rotatebox{90}{$\in$} &                 & \rotatebox{90}{$\in$} &                                  \\
					          & x                     & \longmapsto     & z                     & =g\circ f\qty(x)=g\qty(f\qty(x))
				\end{array}
			\end{align*}
	\end{itemize}

	\subsection{集合の分割と同値関係}
		一般に集合$S$を交わりのない(空でない)集合の和で表すことを\redunderline{類別(classification)}という%undlineは赤線
		\[S=A_1\sqcup A_2\sqcup\cdots\]
		各部分集合$A_j$をこの類別における\redunderline{類}という

		全ての類から一つずつ元を選んできて得られる集合$R$を\redunderline{完全代表系}という.集合$R$の元を\redunderline{代表元}という.
		このとき,任意の$x\in S$に対して,集合$S$から集合族$\qty{A_1,A_2,\cdots}$への\redunderline{自然な全射写像}$x\mapsto A_j\qty(\ni x)$が得られる.

		\begin{dfn}[同値関係]
			集合$S$上の同値関係$\sim\iff$集合$S$上の二項関係$\sim$が任意の$a,b,c\in S$に対し,
			\begin{itemize}
				\item[反射則] $a\sim a$
				\item[対称則] $a\sim b \Rightarrow b\sim a$
				\item[推移則] $a\sim b\text{かつ}b\sim c\Rightarrow a\sim c$
			\end{itemize}
		\end{dfn}

		\begin{dfn}[同値類]
			集合$S$に同値関係$\sim$が存在するとき,$S$の各元$a$に対して定まる空でない部分集合
			\[C_a=\qty{x\mid x\in S,x\sim a}\]
			を$a$の同値類(equivalence class)という.
		\end{dfn}

		\begin{dfn}[商集合]
			同値関係$\sim$による同値類の集合を$S$の$\sim$による商集合(quotient set)といい$S/{\sim}$で表す
			\[S/{\sim}\,=\qty{C_a\mid a\in S}\]
		\end{dfn}

		\begin{dfn}[同値類別]
			同値関係による集合の分割.同値関係$\sim$による同値類別
			\[S=\bigsqcup_{a\in R\subset S}C_a\]
			ここで$S$の部分集合$R$は全ての同値類の代表元の集合であり,完全代表系である.
			このとき,自然な全射$\pi:S\to S/{\sim};x\mapsto C_x$が存在する.
			さらに全単射写像$R\to S/{\sim}$も存在する.
		\end{dfn}
		\begin{screen}
			\[C_a\cap C_b\neq\emptyset\Rightarrow C_a=C_b\]
			が成立する(同値関係の性質から直ちに示される)
		\end{screen}

		\begin{ex}
			$S$:ある学校の学生全体の集合.$a,b\in S$の間の関係「クラスメートである」は同値関係.
			この関係を$\sim^\prime$で表すと,$a\sim^\prime b$は「$a$は$b$とクラスメートである」の意.このとき$C_a=\qty{x\in S\mid x\sim a}$は「$a$さんのクラスメート($a$さんを含む)の全体」$\leftrightarrow$ $a$さんの属するクラス
		\end{ex}
\end{document}
