\documentclass[main]{subfiles}

\usepackage{subfiles}

\begin{document}
\section{準同型定理}
	\redunderline{準同型写像}は,1対1の写像とも,上への写像とも限らない.しかし,その中から\redunderline{同型写像}を取り出すことができる.
	\begin{prop}
		$G$を群.$N\triangleleft G$とする.このとき,標準射影(自然な全射)
		\[\kappa:G\to G/N;a\mapsto\kappa\qty(a)=C_N\qty(a)=aN\]
		は準同型写像となる.
	\end{prop}
	\begin{proof}
		$a,b\in G$とする.このとき,$\kappa\qty(ab)=\qty(ab)N=aN\cdot bN=\kappa\qty(a)\cdot\kappa\qty(b).$
	\end{proof}
	\begin{prop}
		準同型写像の核は,正規部分群.
	\end{prop}
	\begin{proof}
		$f\in\Hom\qty(G,G^\prime)$とする.
		\begin{itemize}
			\item 部分群であること\\$\Ker f$には少なくとも$G$の単位元が属している.$a,b\in \Ker f$に対して,\begin{align*}
						            & f\qty(ab^{-1})=f\qty(a)f\qty(b)^{-1}=e_{G^\prime}\cdot e_{G^\prime}^{-1}=e_{G^\prime} \\
						\Rightarrow & ab^{-1}\in\Ker f
					\end{align*}
					が成立.よって$\Ker f$は$G$の部分群.
			\item 正規であること\\任意の$g\in G$に対して$a\in\Ker f$ならば,
				\[f\qty(gag^{-1})=f\qty(g)\cdot f\qty(a)\cdot f\qty(g)^{-1}=f\qty(g)\cdot e_{G^\prime}\cdot f\qty(g)^{-1}=e_{G^\prime}\]
				より$gag^{-1}\in\Ker f$が成り立つ.
		\end{itemize}
	\end{proof}
	\begin{thm}[準同型定理]
		$f\in\Hom\qty(G,G^\prime)$,$f$の像$\Im f$のなす$G^\prime$の部分群について,
		\[\bar{f}:G/\Ker f\to\Im f;g\Ker f\mapsto f\qty(g)\]
		による,群の同型\redunderline{$G/\Ker f\simeq\Im f$}が成り立つ\footnote{$\kappa$:準同型,$\Ker f$:$G$の正規部分群}.この$\bar{f}$を\redunderline{$f$から誘導される(引き起こされる)同型写像}という.
	\end{thm}
	\begin{align*}
		\begin{array}{ccccc}
			G &                & \stackrel{f}{\longrightarrow} &                                                           & \Im f\subset G^\prime \\
			  & \kappa\searrow &                               & \rotatebox{45}{$\stackrel{\sim}{\longrightarrow}$}\bar{f} &                       \\
			  &                & G/\Ker f                      &                                                           &
		\end{array}
	\end{align*}
	\begin{align*}
		\begin{array}{rccc}
			\bar{f}: & G/\Ker f              & \longrightarrow & \Im f                 \\
			         & \rotatebox{90}{$\in$} &                 & \rotatebox{90}{$\in$} \\
			         & g\Ker f               & \longmapsto     & f\qty(g)
		\end{array}
	\end{align*}
	\begin{proof}
		\begin{itemize}
			\item 写像の定義が代表元によらないこと\\$g^\prime\in g\Ker f$より,
				\begin{align*}
					g^\prime        & =gn\quad\qty(n\in\Ker f)                   \\
					f\qty(g^\prime) & =f\qty(gn)=f\qty(g)\cdot f\qty(n)=f\qty(g)
				\end{align*}
			\item 準同型
				\begin{align*}
					\bar{f}\qty(g\Ker f\cdot g^\prime\Ker f) & =\bar{f}\qty(\qty(gg^\prime)\Ker f)                   \\
					                                         & =f\qty(gg^\prime)=f\qty(g)\cdot f\qty(g^\prime)       \\
					                                         & =\bar{f}\qty(g\Ker f)\cdot\bar{f}\qty(g^\prime\Ker f)
				\end{align*}
			\item 単射\\$\bar{f}\qty(g\Ker f)=\bar{f}\qty(g^\prime\Ker f)$,つまり,$f\qty(g)=f\qty(g^\prime)$とすると,
					\[f\qty(g^{-1}\cdot g^\prime)=f\qty(g)^{-1}\cdot f\qty(g^\prime)=e_{G^\prime}\]
					となり,$g^{-1}\cdot g^\prime\in\Ker f$である.よって$g^\prime\in g\Ker f$から,
				\[g\Ker f=g^\prime\Ker f\]
				が得られる.
			\item 全射\\任意の$\tilde{g}\in\Im f$から$\bar{f}\qty(g\Ker f)=\tilde{g}$を満たす$g\Ker f\in G/\Ker f$が存在することを示す.$\Im f$の定義より,$\forall\tilde{g}\in\Im f$に対して$f\qty(g)=\tilde{g}$を満たす$g\in G$が存在する.この$g$の属する剰余集合$g\Ker f$を取れば,$\bar{f}\qty(g\Ker f)=f\qty(g)=\tilde{g}$である.
		\end{itemize}
	\end{proof}
	\begin{cor}
		特に$G$が有限群のとき,
		\[\qty|\Im f|=\qty|G|/\qty|\Ker f|\]
		が成り立つ.
	\end{cor}
	\begin{note*}
		線形代数の線形写像に関する次元定理に相当する.
	\end{note*}
	\begin{ex}[対称群$\frakS_n$と交代群$\frakA_n$]
		$\frakS_n$から積を演算とする$\bbR^\times\coloneqq\bbR\backslash\qty{0}$への写像
		\[\sgn:\frakS_n\to\bbR^\times;\sigma\mapsto\sgn\sigma=\begin{cases}
				+1 & \sigma:\text{偶置換} \\
				-1 & \text{それ以外}
			\end{cases}\]
		は準同型写像であり,$\Ker\sgn=\frakA_n,\Im\sgn=\qty{+1,-1}$である.
		\[\bar{f}:\frakS_n/\frakA_n\to\qty{\pm 1};\sigma\frakA_n\mapsto\sgn\sigma\]
		例えば,$\sigma_1=\qty(1, 2)$とすると
		\[\frakS_n/\frakA_n=\qty{e\frakA_n,\sigma_1\frakA_n}\]
		となり,
		\[\bar{f}\qty(e\frakA_n)=1,\bar{f}\qty(\sigma_1\frakA_n)=-1\]
		これは,$\qty{\pm 1}$への同型写像.
	\end{ex}
	\begin{ex}
		$\GL\qty(n;\bbC)$から$\bbC^\times\coloneqq\bbC\backslash\qty{0}$への写像$\det:A\mapsto\det A$は準同型であり,
		\begin{align*}
			\Ker\det & =\SL\qty(n;\bbC)=\qty{g\in \GL\qty(n,\bbC)\mid\det g=1} \\
			\Im\det  & =\bbC^\times=\bbC\backslash\qty{0}
		\end{align*}
		である.
		\[\GL\qty(n;\bbC)/\SL\qty(n;\bbC)\simeq\bbC^\times\]
	\end{ex}
\end{document}
