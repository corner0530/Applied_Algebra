\documentclass[main]{subfiles}

\usepackage{subfiles}

\begin{document}
\section{表現の指標と直交性}
	\begin{dfn}[指標(character)]
		有限群$G$の$d$次元表現$\qty(\rho,V)$に対し,指標と呼ばれる$G$上の関数$\chi^\rho$を
		\[\chi^\rho\qty(g)=\chi\qty(g)=\Trace\qty(\rho\qty(g))=\sum_{i=1}^d\rho_{ii}\qty(g)\quad\qty(g\in G)\]
		で定める.このとき,単位元の表現$\rho\qty(e)$は単位行列であり,$\chi\qty(e)=d$.
	\end{dfn}
	\begin{ex}[$D_3$の$\rho_\text{ex5}$の指標]
		\[\chi\qty(e)=3,\chi\qty(a)=\chi\qty(a^2)=0,\chi\qty(b)=\chi\qty(ab)=\chi\qty(ba)=1\]
	\end{ex}
	\begin{note*}
		$D_3$の共役類\footnote{$g\stackrel{\text{conj}}{\sim}\suchthat\exists a\in G,ag^\prime a^{-1}=g$}
		\[C_1=\qty{e},C_2=\qty{a,a^2},C_3=\qty{b,ab,ba}\]
	\end{note*}
	\begin{dfn}[類関数]
		群$G$上の関数$\varphi$が類関数であるとは,$G$の任意の元$g,h$に対して,
		\[\varphi\qty(ghg^{-1})=\varphi\qty(h)\]
		が成り立つこと.このとき,$\varphi\qty(gh)=\varphi\qty(hg)$.
	\end{dfn}
	\begin{cor}
		同値な表現$\rho$と$\rho^\prime$に対する指標は等しい.また,共役な言,すなわち,同一共役類の指標は等しい.指標は類関数の1つ.
	\end{cor}
	\begin{proof}
		$T\rho T^{-1}=\rho^\prime$ならば,
		\[\Tr\qty(\rho^\prime)=\Tr\qty(T\rho T^\prime)=\Tr\qty(\rho)\]
		$gg_ig^{-1}=g_j$ならば,
		\[\Tr\qty(\rho\qty(g_j))=\Tr\qty(\rho\qty(g)\rho\qty(g_i)\rho\qty(g)^{-1})=\Tr\qty(\rho\qty(g_i))\]
	\end{proof}
	\begin{thm}
		$G$の表現$\lambda$と$\mu$が同値であるためには,その指標$\chi^\lambda$と$\chi^\mu$が$G$上の関数として一致することが必要十分である.
	\end{thm}
	証明は後述.
	\begin{cor}
		群$G$の$d$次元表現$\qty(\rho,V)$の指標$\chi$は次の性質を持つ.
		\begin{enumerate}
			\item $\chi\qty(e)=d$
			\item $\chi\qty(g)$は$d$個の1の$\qty|G|$乗根の和
			\item $\chi\qty(g^{-1})=\overline{\chi\qty(g)}$
		\end{enumerate}
	\end{cor}
	\begin{proof}
		\begin{enumerate}
			\item $\rho\qty(e)=\text{Id}$から明らか.
			\item $\rho\qty(g)$の固有値を重複を含め$\lambda_1,\cdots,\lambda_d$とし,対応する固有ベクトル$v_1,\cdots,v_d$とする.このとき,$\chi\qty(g)=\lambda_1+\cdots+\lambda_d$であり,$\rho\qty(g)v_i=\lambda_iv_i$を満たす.さらに\footnote{$g^\qty|G|=e$}\[\lambda_i^\qty|G|v_i=\rho\qty(g^\qty|G|)v_i=v_i\]よって$\lambda_i$は1の$\qty|G|$乗根,ここでゼロ固有値でもないことに注意.
			\item ユニタリー表現から,\[\chi\qty(g^{-1})=\Tr\qty(\rho\qty(g^{-1}))=\Tr\qty(^{\top}\rho\qty(g^{-1}))=\Tr\qty(\overline{\rho}\qty(g))=\overline{\chi}\qty(g)\]$G$上の複素数値関数$\varphi,\psi$に対する内積を次で定義する.\[\left<\varphi,\psi\right>=\frac{1}{\qty|G|}\sum_{g\in G}\varphi\qty(g)\overline{\psi\qty(g)}\]指標の性質から,指標の内積\[\left<\chi^\lambda,\chi^\mu\right>=\frac{1}{\qty|G|}\sum_{g\in G}\chi^\lambda\qty(g)\chi^\mu\qty(g^{-1})\]
		\end{enumerate}
	\end{proof}
	\begin{lem}[Schurの補題1]
		$F=\bbR$または$\bbC$とする,群$G$の$F$上の2つの表現,$m$次既約表現$\rho^\qty(1)$と$n$次既約表現$\rho^\qty(2)$に対して,$F$上の$m\times n$行列$M$が
		\[\rho^\qty(1)\qty(g)M=M\rho^\qty(2)\qty(g)\]
		を満たすならば,$M$は正則行列(このとき$m=n$)であるか,または$M=O$である.
	\end{lem}
	\begin{proof}
		任意の$x\in\Ker M$に対して
		\[M\qty(\rho^\qty(2)\qty(g)x)=\rho^\qty(1)\qty(g)Mx=0\]
		よって$\rho^\qty(2)\qty(g)x\in\Ker M$であり,$\Ker M$は$\rho^\qty(2)$の不変部分空間.さらに$\rho^\qty(2)$は既約なので,不変部分空間である$\Ker M$は零空間$\qty{0}$あるいは全空間$F^n$に等しい.

		また,任意の$y\in\Im M$に対して,$y=Mx$なる$x$によって,
		\[\rho^\qty(1)\qty(g)y=\rho^\qty(1)\qty(g)Mx=M\qty(\rho^\qty(2)\qty(g)x)\in\Im M\]
		が従う.よって$\Im M$は$\rho^\qty(1)$に関する不変部分空間であり,$\rho^\qty(1)$が既約表現であることより,$\Im M$は零空間$\qty{0}$あるいは全空間$F^m$に等しい.

		したがって$M$は零行列あるいは正則行列.
	\end{proof}
\end{document}
