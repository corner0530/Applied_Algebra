\documentclass[main]{subfiles}

\usepackage{subfiles}

\begin{document}
\section{表現の指標と直交性}
	\begin{dfn}[指標(character)]
		有限群$G$の$d$次元表現$\qty(\rho,V)$に対し,指標と呼ばれる$G$上の関数$\chi^\rho$を
		\[\chi^\rho\qty(g)=\chi\qty(g)=\Trace\qty(\rho\qty(g))=\sum_{i=1}^d\rho_{ii}\qty(g)\quad\qty(g\in G)\]
		で定める.このとき,単位元の表現$\rho\qty(e)$は単位行列であり,$\chi\qty(e)=d$.
	\end{dfn}
	\begin{ex}[$D_3$の$\rho_\text{ex5}$の指標]
		\[\chi\qty(e)=3,\chi\qty(a)=\chi\qty(a^2)=0,\chi\qty(b)=\chi\qty(ab)=\chi\qty(ba)=1\]
	\end{ex}
	\begin{note*}
		$D_3$の共役類\footnote{$g\stackrel{\text{conj}}{\sim}\suchthat\exists a\in G,ag^\prime a^{-1}=g$}
		\[C_1=\qty{e},C_2=\qty{a,a^2},C_3=\qty{b,ab,ba}\]
	\end{note*}
	\begin{dfn}[類関数]
		群$G$上の関数$\varphi$が類関数であるとは,$G$の任意の元$g,h$に対して,
		\[\varphi\qty(ghg^{-1})=\varphi\qty(h)\]
		が成り立つこと.このとき,$\varphi\qty(gh)=\varphi\qty(hg)$.
	\end{dfn}
	\begin{cor}
		同値な表現$\rho$と$\rho^\prime$に対する指標は等しい.また,共役な言,すなわち,同一共役類の指標は等しい.指標は類関数の1つ.
	\end{cor}
	\begin{proof}
		$T\rho T^{-1}=\rho^\prime$ならば,
		\[\Tr\qty(\rho^\prime)=\Tr\qty(T\rho T^\prime)=\Tr\qty(\rho)\]
		$gg_ig^{-1}=g_j$ならば,
		\[\Tr\qty(\rho\qty(g_j))=\Tr\qty(\rho\qty(g)\rho\qty(g_i)\rho\qty(g)^{-1})=\Tr\qty(\rho\qty(g_i))\]
	\end{proof}
	\begin{thm}
		$G$の表現$\lambda$と$\mu$が同値であるためには,その指標$\chi^\lambda$と$\chi^\mu$が$G$上の関数として一致することが必要十分である.
	\end{thm}
	証明は後述.
	\begin{cor}
		群$G$の$d$次元表現$\qty(\rho,V)$の指標$\chi$は次の性質を持つ.
		\begin{enumerate}
			\item $\chi\qty(e)=d$
			\item $\chi\qty(g)$は$d$個の1の$\qty|G|$乗根の和
			\item $\chi\qty(g^{-1})=\overline{\chi\qty(g)}$
		\end{enumerate}
	\end{cor}
	\begin{proof}
		\begin{enumerate}
			\item $\rho\qty(e)=\text{Id}$から明らか.
			\item $\rho\qty(g)$の固有値を重複を含め$\lambda_1,\cdots,\lambda_d$とし,対応する固有ベクトル$v_1,\cdots,v_d$とする.このとき,$\chi\qty(g)=\lambda_1+\cdots+\lambda_d$であり,$\rho\qty(g)v_i=\lambda_iv_i$を満たす.さらに\footnote{$g^\qty|G|=e$}\[\lambda_i^\qty|G|v_i=\rho\qty(g^\qty|G|)v_i=v_i\]よって$\lambda_i$は1の$\qty|G|$乗根,ここでゼロ固有値でもないことに注意.
			\item ユニタリー表現から,\[\chi\qty(g^{-1})=\Tr\qty(\rho\qty(g^{-1}))=\Tr\qty(^{\top}\rho\qty(g^{-1}))=\Tr\qty(\overline{\rho}\qty(g))=\overline{\chi}\qty(g)\]$G$上の複素数値関数$\varphi,\psi$に対する内積を次で定義する.\[\left<\varphi,\psi\right>=\frac{1}{\qty|G|}\sum_{g\in G}\varphi\qty(g)\overline{\psi\qty(g)}\]指標の性質から,指標の内積\[\left<\chi^\lambda,\chi^\mu\right>=\frac{1}{\qty|G|}\sum_{g\in G}\chi^\lambda\qty(g)\chi^\mu\qty(g^{-1})\]
		\end{enumerate}
	\end{proof}
	\begin{lem}[Schurの補題1]
		$F=\bbR$または$\bbC$とする,群$G$の$F$上の2つの表現,$m$次既約表現$\rho^\qty(1)$と$n$次既約表現$\rho^\qty(2)$に対して,$F$上の$m\times n$行列$M$が
		\[\rho^\qty(1)\qty(g)M=M\rho^\qty(2)\qty(g)\]
		を満たすならば,$M$は正則行列(このとき$m=n$)であるか,または$M=O$である.
	\end{lem}
	\begin{proof}
		任意の$x\in\Ker M$に対して
		\[M\qty(\rho^\qty(2)\qty(g)x)=\rho^\qty(1)\qty(g)Mx=0\]
		よって$\rho^\qty(2)\qty(g)x\in\Ker M$であり,$\Ker M$は$\rho^\qty(2)$の不変部分空間.さらに$\rho^\qty(2)$は既約なので,不変部分空間である$\Ker M$は零空間$\qty{0}$あるいは全空間$F^n$に等しい.

		また,任意の$y\in\Im M$に対して,$y=Mx$なる$x$によって,
		\[\rho^\qty(1)\qty(g)y=\rho^\qty(1)\qty(g)Mx=M\qty(\rho^\qty(2)\qty(g)x)\in\Im M\]
		が従う.よって$\Im M$は$\rho^\qty(1)$に関する不変部分空間であり,$\rho^\qty(1)$が既約表現であることより,$\Im M$は零空間$\qty{0}$あるいは全空間$F^m$に等しい.

		したがって$M$は零行列あるいは正則行列.
	\end{proof}
	\begin{lem}[Schurの補題2]
		群$G$の$\bbC$上の有限次元表現$\rho$が既約ならば,全ての元$g\in G$に対して\redunderline{$\rho\qty(g)M=M\rho\qty(g)$}を満たす行列は$M=\lambda I$に限られる.ただし,$\lambda\in\bbC$,$I$は単位行列.
	\end{lem}
	\begin{proof}
		$\bbC$上なので$M$の固有値$\lambda\in\bbC$が存在し,
		\[\rho\qty(g)\qty(M-\lambda I)=\qty(M-\lambda I)\rho\qty(g),\ g\in G\]
		ここで$M-\lambda I$は正則でなく,$\rho\qty(g)$は既約であることから,Schurの補題1より,$M-\lambda I=O$
	\end{proof}
	\begin{remark*}
		無限行列では固有値を持たないこともある.例えば,$\qty(a_1,a_2,\cdots)\mapsto\qty(0,a_1,a_2,\cdots)$
	\end{remark*}
	\begin{thm}[表現の直交性定理]
		位数$\qty|G|$の群$G$の2つの$\bbC$上の既約表現$\rho^\qty(\alpha)\qty(g),\rho^\qty(\beta)\qty(g)$の行列要素は次の直交関係を満たす.
		\[\sum_{g\in G}\rho^\qty(\alpha)_{ji}\qty(g)\cdot\rho^\qty(\beta)_{kl}\qty(g^{-1})=\frac{\qty|G|}{d_\alpha}\delta_{ik}\delta_{jl}\delta_{\alpha\beta}\]
		ここで,$d_\alpha,d_\beta$はそれぞれ表現$\rho^\qty(\alpha)$と$\rho^\qty(\beta)$の次元.
		\[\delta_{\alpha\beta}=\begin{cases}
				1 & \qty(\rho^\qty(\alpha)\text{と}\rho^\qty(\beta)\text{が同一既約表現})     \\
				0 & \qty(\rho^\qty(\alpha)\text{と}\rho^\qty(\beta)\text{が同値な既約表現でない})
			\end{cases}\]
	\end{thm}
	\begin{proof}
		任意の行列$B\in\Mat\qty(d_\alpha\times d_\beta)$を用いて,行列$M\in\Mat\qty(d_\alpha\times d_\beta):\,M=\sum_{g\in G}\rho^\qty(\alpha)\qty(g)B\rho^\qty(\beta)\qty(g^{-1})$を導入し,任意の元$g,g^\prime\in G$に対し,$g^\prime g=g^{\prime\prime}$なる群$G$の元$gg^{\prime\prime}$が存在することを用いると,
		\begin{align*}
			\rho^\qty(\alpha)\qty(g^\prime)M & =\sum_{g\in G}\rho^\qty(\alpha)\qty(g^\prime)\rho^\qty(\alpha)\qty(g)B\rho^\qty(\beta)\qty(g^{-1}) \\
			                                 & =\sum_{g\in G}\rho^\qty(\alpha)\qty(g^\prime g)B\rho^\qty(\beta)\qty(g^{-1})                       \\
			                                 & =\sum_{g\in G}\rho^\qty(\alpha)\qty(g^{\prime\prime})B\rho^\qty(\beta)\qty(g^{-1})
		\end{align*}
		が得られ,$g$に関する和を$g^{\prime\prime}$に関する和に書き換えると,
		\begin{align*}
			\rho^\qty(\alpha)\qty(g^\prime)M & =\sum_{g^{\prime\prime}\in G}\rho^\qty(\alpha)\qty(g^{\prime\prime})B\rho^\qty(\beta)\qty({g^{\prime\prime}}^{-1}g^\prime)                       \\
			                                 & =\sum_{g^{\prime\prime}\in G}\rho^\qty(\alpha)\qty(g^{\prime\prime})B\rho^\qty(\beta)\qty({g^{\prime\prime}}^{-1})\rho^\qty(\beta)\qty(g^\prime) \\
			                                 & =M\rho^\qty(\beta)\qty(g^\prime)
		\end{align*}
		\begin{itemize}
			\item \underline{$\alpha\neq\beta$の場合}\\
				Schurの補題1から$M=O$.行列$B$を$B_{ik}=1$,それ以外は0と選ぶと\[\sum_{g\in G}\rho_{ji}^\qty(\alpha)\qty(g)\rho_{kl}^\qty(\beta)\qty(g^{-1})=0\]である.
			\item \underline{$\alpha=\beta$の場合}(同一の既約表現)\\
				Schurの補題2から$M=\lambda I$.行列$B$を$B_{ik}=1$,それ以外を0とする.両辺の$\qty(j,l)$要素から,\[\lambda\delta_{jl}=\sum_{g\in G}\rho_{ji}^\qty(\alpha)\qty(g)\rho^\qty(\alpha)_{kl}\qty(g^{-1})\]差に,$j=l$と置いて$j$について和を取ると\begin{align*}
					\lambda d_\alpha & =\sum_{g\in G}\sum_{j=1}^{d_\alpha}\rho_{kj}^\qty(\alpha)\qty(g^{-1})\rho_{ji}^\qty(\alpha)\qty(g) \\
					                 & =\qty|G|\delta_{ki}
				\end{align*}が$\rho^\qty(\alpha)\qty(g^{-1})\rho^\qty(\alpha)\qty(g)=I_{d_\alpha}$より導かれる.定数$\lambda=\qty|G|\delta_{ki}/d_\alpha$である.
		\end{itemize}
	\end{proof}
	\begin{thm}[指標の第1種直交性]
		群$G$の$\bbC$上の規約指標$\chi^\qty(\alpha),\chi^\qty(\beta)$に対して$\left<\chi^\qty(\alpha),\chi^\qty(\beta)\right>=\delta^\prime_{\alpha\beta}$が成り立つ.

		(別表示)\[\sum_{i=1}^{n_c}\qty|C_i|\chi^\qty(d)\qty(C_i)\overline{\chi}^\qty(\beta)\qty(C_i)=\qty|G|\delta^\prime_{\alpha\beta}\]
		ただし,共役類$C_i$の元の指標$\chi\qty(C_i)$,群$G$の共役類の個数を$n$とする.
	\end{thm}
	\begin{proof}
		表現の直交性定理において$i=j,k=l$として,$k,i$についてそれぞれ和を取る.
		\begin{align*}
			\sum_g\rho_{ii}^\qty(\alpha)\qty(g)\overline{\chi}^\qty(\beta)\qty(g) & =\sum_{k=1}^{d_\beta}\frac{\qty|G|}{d_\alpha}\delta_{ik}\delta_{\alpha\beta}=\frac{\qty|G|}{d_\alpha}\delta_{\alpha\beta} \\
			\sum_g\chi^\qty(\alpha)\qty(g)\overline{\chi}^\qty(\beta)\qty(g)      & =\sum_{i=1}^{d_\alpha}\frac{\qty|G|}{d_\alpha}\delta_{\alpha\beta}=\qty|G|\delta_{\alpha\beta}
		\end{align*}
		指標は同値な表現に対して同じ値を取るので
		\[\delta^\prime_{\alpha\beta}=\begin{cases}
				1 & \qty(\alpha\text{と}\beta\text{が同値な既約表現}) \\
				0 & \qty(\text{それ以外})
			\end{cases}\]
		(別表示については,同じ共役類に属する元の指標は等しい.)
	\end{proof}
	\subparagraph{$\bbC$上の既約表現の直和による分解}
		一般な可約表現$\rho$は,適当なユニタリ変換によって既約表現の直和に分解できる.
		\[\rho\qty(g)\sim\bigoplus_{\alpha=1}^{n_r}\bigoplus_{i=1}^{q_\alpha}\rho^\qty(\alpha)\qty(g)\]
		\begin{itemize}
			\item $\bigoplus_\alpha$:群$G$の全ての$\bbC$上の既約表現$\rho^\qty(\alpha)$に関する直和
			\item $q_\alpha\in\bbZ_{\geq 0}$: $\rho^\qty(\alpha)$の多重度
			\item 群の同値でない既約表現の全体$\qty{\rho^\qty(1),\cdots,\rho^\qty(n_r)}$
		\end{itemize}
		表現$\rho$の指標$\chi\qty(g)$と$\rho^\qty(\alpha)\qty(g)$の既約指標$\chi^\qty(\alpha)\qty(g)$を用いることで$\chi\qty(g)=\sum q_\alpha\chi^\qty(\alpha)\qty(g)$が$A$のtraceから得られる.さらに両辺に$\overline{\chi}^\qty(\beta)\qty(g)$をかけて元$g$について和を取ると指標の直交性から,
		\[q_\beta=\frac{1}{\qty|G|}\sum_{g\in G}\chi\qty(g)\overline{\chi}^\qty(\beta)\qty(g)\]
		あるいは
		\[q_\beta=\frac{1}{\qty|G|}\sum_{i=1}^{n_C}\qty|C_i|\chi\qty(C_i)\overline{\chi}^\qty(\beta)\qty(C_i)\]
		が得られる.すなわち,既約指標$\chi^\qty(\alpha)$が分かれば,群の表現$\rho$の中に既約表現$\rho^\qty(\alpha)$の多重度$q_\alpha$が得られる.
	\subparagraph{$D_3$の指標と既約分解}
		\[D_3=\qty{e,a,a^2,b,ab,ba},C_1=\qty{e},C_2=\qty{a,a^2},C_3=\qty{b,ab,ba}\]
		既約表現$\rho^\qty(\text{ex1}),\rho^\qty(\text{ex2}),\rho^\qty(\text{ex3})$に関する各共役類の指標
		\begin{align*}
			\begin{array}{lll}
				\chi^\qty(\text{ex1})\qty(C_1)=1, & \chi^\qty(\text{ex1})\qty(C_2)=1,  & \chi^\qty(\text{ex1})\qty(C_3)=1  \\
				\chi^\qty(\text{ex2})\qty(C_1)=1, & \chi^\qty(\text{ex2})\qty(C_2)=1,  & \chi^\qty(\text{ex2})\qty(C_3)=-1 \\
				\chi^\qty(\text{ex3})\qty(C_1)=2, & \chi^\qty(\text{ex3})\qty(C_2)=-1, & \chi^\qty(\text{ex3})\qty(C_3)=0
			\end{array}
		\end{align*}

		(可約な)表現$\rho_\text{ex5}$の各共役類の指標
		\[\chi\qty(C_1)=3,\chi\qty(C_2)=0,\chi\qty(C_3)=1\]

		(可約な)表現$\rho_\text{ex5}$に含まれる既約表現の多重度
		\begin{align*}
			\rho_\text{ex1} & =\frac{1}{6}\qty(\qty|C_1|\chi\qty(C_1)\overline{\chi}^\qty(\text{ex1})\qty(C_1)+\cdots+\qty|C_3|\chi\qty(C_3)\overline{\chi}^\qty(\text{ex1})\qty(C_3)) \\
			                & =\frac{1}{6}\qty(1\times 3\times 1+2\times 0\times 1+3\times 1\times 1)=1                                                                                \\
			\rho_\text{ex2} & =\frac{1}{6}\qty(1\times 3\times 1+2\times 0\times 1+3\times 1\times\qty(-1))=0                                                                          \\
			\rho_\text{ex3} & =\frac{1}{6}\qty(1\times 3\times 2+2\times 0\times\qty(-1)+3\times 1\times 0)=1                                                                          \\
		\end{align*}
		以上より,ブロック対角化を経由せずに
		\[\rho=\rho^\qty(\text{ex1})\oplus\rho^\qty(\text{ex3})\]
		が示された.
\end{document}
