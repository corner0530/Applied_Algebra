\documentclass[main]{subfiles}

\usepackage{subfiles}

\begin{document}
\section{準同型写像}
	準同形写像: 群の二項演算の代数的構造を保つ写像
	\subsubsection*{群の例}
		\begin{enumerate}
			\item[(1)] $G=\bbZ,\quad a\circ b=a+b+1,\quad\qty(G,\circ)$は群をなす.この群の単位元および$a$の逆元を求めよ
				\[a\circ e=e\circ a=a\iff a+e+1=e+a+1=a\iff e=-1\]
				\[a\circ a^{-1}=a^{-1}\circ a=e\iff a+a^{-1}+1=a^{-1}+a+1=-1\iff a^{-1}=-a-2\]
			\item[(2)] $G=\bbR\backslash\qty{-1},\quad a\circ b=a+b+ab,\quad\qty(G,\circ)$は群をなす.この群の単位元および$a$の逆元を求めよ
				\[e=0,\quad a^{-1}=-\frac{a}{a+1}\]
			\item[(3)] $G=\bbR\backslash\qty{0},\quad a\circ b=2ab,\quad\qty(G,\circ)$は群をなす.この群の単位元および$a$の逆元を求めよ
				\[e=\frac{1}{2},\quad a^{-1}=\frac{1}{4a}\]
		\end{enumerate}
		これらの例では,より簡単な群と関係づけることができる.以下,準同型,同型を導入する.
		\begin{dfn}
			2つの群$G,G^\prime$に対して,写像$f:G\to G^\prime$が
			\[\redunderline{f\qty(x\,{\color{blue}\circ}\,y)=f\qty(x)\,{\color{blue}\cdot}\,f\qty(y)\quad\qty(x,y\in G)}\]
			を満たすとき,$f$を\redunderline{準同型写像}または\redunderline{準同型(homomorphism)}という.また,$G$から$G^\prime$への準同型写像全体を$\Hom\qty(G,G^\prime)$とかく.
		\end{dfn}
		\begin{remark*}
			$G,G^\prime$として,$\qty(G,\circ),\qty(G^\prime,\cdot)$を意味している.
		\end{remark*}
		\begin{itemize}
			\item $G$上の二項演算$\circ$を$G^\prime$上の二項演算$\cdot$へ移す.
				\[f\qty(x\circ y)=f\qty(x)\cdot f\qty(y)=x^\prime\cdot y^\prime\quad\text{ここで}x^\prime,y^\prime\in G^\prime\]
			\item 群$G$の単位元$e$は,群$G^\prime$の単位元$e^\prime$に移される.
				\begin{align*}
					     & f\qty(e)\cdot f\qty(e)=f\qty(e\circ e)=f\qty(e)                 \\
					\iff & f\qty(e)^{-1}\cdot f\qty(e)\cdot f\qty(e)=f\qty(e)^{-1}f\qty(e) \\
					\iff & \,e^\prime\cdot f\qty(e)=f\qty(e)=e^\prime
				\end{align*}
			\item 群$G$の逆元は,群$G^\prime$の逆元に移される.
				\begin{align*}
					     & f\qty(a)\cdot f\qty(a^{-1})=\qty(a\circ a^{-1})=f\qty(e)=e^\prime \\
					\iff & f\qty(a^{-1})=f\qty(a)^{-1}
				\end{align*}
				\begin{align*}
					\begin{array}{ccc}
						\qty(x,y)\in G^2                        & \stackrel{\circ}{\longmapsto} & x\circ y\in G                      \\
						\downarrow f                            &                               & \downarrow f                       \\
						\qty(f\qty(x),f\qty(y))\in G^{\prime 2} & \longmapsto                   & f\qty(x)\cdot f\qty(y)\in G^\prime
					\end{array}
				\end{align*}
		\end{itemize}
		\begin{dfn}
			準同型$f$が全単射のとき,$f$を\redunderline{同型写像}または\redunderline{同型(isomorphism)}という.同型写像$f:G\to G^\prime$が存在すれば\redunderline{$G$と$G^\prime$は同型}であるといい.$G\simeq G^\prime$または$G\stackrel{\sim}{\longrightarrow}G^\prime$などとかく.
		\end{dfn}
		\begin{ex}
			同型を用いると,先の例(1),(2),(3)は\dots

			(1)の群は,通常の加法に関する$\bbZ$のなす群と同型.$f\qty(a)=a+1$とする.
			\[f\qty(a\circ b)=f\qty(a+b+1)=a+b+2=\qty(a+1)+\qty(b+1)=f\qty(a)+f\qty(b)\]

			(2)と(3)の群は,通常の乗法群としての$\bbR^\times=\bbR\backslash\qty{0}$と同型
			\begin{itemize}
				\item[(2)] $f^\prime\qty(a)=a+1$
				\item[(3)] $f^\prime\qty(a)=2a$
			\end{itemize}
			{\color{red} 準同型,全単射であることは容易に確かめられる.}
		\end{ex}
	\subsubsection*{準同型の例}
		1次分数変換
		\[G=\qty{f\qty(x)=\frac{ax+b}{cx+d}\mid a,b,c,d\in\bbC,ad-bc\neq 0}\]
		$\GL\qty(2,\bbC)$\footnote{$\qty|A|=ad-bc\neq 0$}の元$A=\mqty[
				a & b \\
				c & d
			]$に対して,$G$の元$f_A\qty(x)$を$f_A\qty(x)=\frac{ax+b}{cx+d}$によって定める.
		$\GL\qty(2,\bbC)$の元と$G$の元を対応\footnote{$F:\GL\qty(2,\bbC)\to G$}付けている.
		\[f_{\scriptsize\mqty[
						1 & 0 \\
						0 & 1
					]}\qty(x)=\frac{x+0}{0+1}=x,\quad f_A^{-1}\qty(f_A\qty(x))=x\]
		\[f_A^{-1}\qty(x)=\frac{1}{ad-bc}\frac{dx-b}{-cx+a}=f_{A^{-1}}\qty(x)\]
		ここで$A^{-1}=\frac{1}{ad-bc}\mqty[
				d  & -b \\
				-c & a
			]$であることに注意.

		このとき,任意の$A=\mqty[
				a & b \\
				c & d
			],\quad P=\mqty[
				p & q \\
				r & s
			]\in \GL\qty(2,\bbC)$に対して,
		\begin{align*}
			\begin{array}{ccc}
				\qty(A,P)     & \longmapsto & A\cdot P  \in \GL\qty(2,\bbC) \\
				F\downarrow   &             & F\downarrow                   \\
				\qty(f_A,f_P) & \longmapsto & f_A\circ f_P=f_{AP}
			\end{array}
		\end{align*}
		{\color{red} 対応$F$は$\GL\qty(2,\bbC)\to G$の準同型を与えている.}
		\begin{remark*}
			$G$は写像の合成に関して群.
		\end{remark*}
		$\GL\qty(2,\bbC)$は行列の積に関して群,単位元は$\mqty[1&0\\0&1]$,$A$の逆行列$A^{-1}$が$A$の逆元.
		ただし,この対応$F$は,全単射となっておらず同型ではない.$\Rightarrow$準同型と同型のズレを測るのが「核」である.
		\begin{dfn}
			準同型$f:G\to G^\prime$に対して,
			\[\Ker f=\qty{x\in G\mid f\qty(x)=e^\prime}\]
			を準同型写像$f$の\redunderline{核(kernel)}という.
		\end{dfn}
		\begin{dfn}
			準同型$f:G\to G^\prime$に対して,
			\[\Im f=\qty{f\qty(x)\mid x\in G}\]
			を準同型写像$f$の\redunderline{像(image)}という.
		\end{dfn}
		\begin{ex}
			\begin{align*}
				\begin{array}{ccc}
					f:\GL\qty(2,\bbC)     & \longrightarrow & G=\qty{\phi\qty(z)=\frac{az+b}{cz+d}\mid a,b,c,d\in\bbC,ad-bc\neq 0} \\
					\rotatebox{90}{$\in$} &                 & \rotatebox{90}{$\in$}                                                \\
					\mqty[
					a                     & b                                                                                      \\
					c                     & d
					]                     & \longmapsto     & \phi_A\qty(z)=\frac{az+b}{cz+d}
				\end{array}
			\end{align*}
			は準同型.核
			\begin{align*}
				\Ker f & =\qty{A\in \GL\qty(2,\bbC)\mid\phi_A\qty(z)}     \\
				       & =\qty{\mqty[a                                & 0 \\0&a]\mid a\in\bbC\backslash\qty{0}}
			\end{align*}
			つまり,分子と分母に定数倍の不定性が存在することを示している.
		\end{ex}
		\begin{prop}
			核が単位元のみからなることが,群の準同型が1対1となるための必要十分条件.
		\end{prop}
		\begin{proof}
			(必要条件は明らか)

			十分性を準同型$f:G\to G^\prime$の場合に示す.

			$f\qty(x)=f\qty(y)$とすると,準同型性から
			\[f\qty(x\circ y^{-1})=f\qty(x)\cdot f\qty(y)^{-1}=e\]
			よって核が単位元のみとすると
			\[x\circ y^{-1}=e\iff x=y\]
		\end{proof}
		\begin{prop}
			$f\in\Hom\qty(G,G^\prime),H\leqq G,H^\prime\leqq G^\prime$とする.
			\begin{itemize}
				\item $f\qty(H)\leqq G^\prime$
				\item $f^{-1}\qty(H^\prime)\leqq G$
			\end{itemize}
			\begin{proof}
				(レポート)
			\end{proof}
		\end{prop}
		\begin{cor}
			自明な群$\qty{e^\prime}\subset G^\prime$の逆像である準同期写像$f$の核$\Ker f$は$G$の部分群となる.
		\end{cor}
		\begin{note*}
			ベクトル空間の間の準同型のことを,通常,線形写像と呼ぶ.

			線形写像$\varphi:V\to W$に対して,$\Ker\varphi$は,$V$の部分ベクトル空間.$\Im\varphi$は,$W$の部分ベクトル空間.
		\end{note*}
		一般の準同型$f:G\to G^\prime$は,$G$の元をいくつかずつまとめて,より``粗い''群を作って$G^\prime$に埋め込む.準同型$f$が元の群をどれだけ``粗く''するかは,$f$の像の1点,特に単位元の逆像のみで決まる.
		\begin{dfn}
			$G$を群とする.$G$から$G$への準同型写像を$G$上の\redunderline{自己準同型(endomorphism)}という.$G$上の自己準同型写像の全体$\End\qty(G)$はモノイドをなす.さらに同型写像である時は,$G$上の\redunderline{自己同型(automorphism)}という.$G$上の自己同型全体からなる集合$\Aut\qty(G)$を\redunderline{自己同型群}という.
		\end{dfn}
		\begin{ex}[全単射だが同型でない例]
			\begin{align*}
				G      & :\,\text{位数}n\text{の有限群} \\
				\phi_a & :\,G\to G;\,g\mapsto ag
			\end{align*}
			ここで$a\in G$とする.\footnote{準同型でないことがわかる}
		\end{ex}
	\subsubsection*{自己同型の重要な例}
		\begin{prop}
			群$G$の元$a\in G$を選ぶ.$a\in G$による\redunderline{共役変換}と呼ばれる写像
			\[A_a:G\to G;\,g\mapsto aga^{-1}\]
			で定義する.これは$G$上の自己同型を与える.
		\end{prop}
		\begin{abbr*}
			\begin{itemize}
				\item 準同型性\\\[A_a\qty(g)A_a\qty(h)=aga^{-1}aha^{-1}=agha^{-1}=A_a\qty(gh)\]
				\item 全単射\ (略)
			\end{itemize}
		\end{abbr*}
\end{document}
